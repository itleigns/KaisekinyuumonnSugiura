\documentclass{jsarticle}
\usepackage{amsmath,amssymb}
\usepackage{bm}
\usepackage[dvipdfmx]{graphicx}
\usepackage{listings,jlisting}
\usepackage{siunitx}
\usepackage{theorem}
\title{解析入門 解答}
\author{itleigns}
\begin{document}
\maketitle{}
\section*{第I章実数と連続}
\subsection*{$\S$1 実数}
問1(i)

$a,b\in K$が両方(R3)を満たす0であると仮定する.

$a$が(R3)を満たす$0$なので
$b+a=b$

$b$も(R3)を満たす$0$なので
$a+b=a$

また(R1)より
$a+b=b+a$


以上より
$a=b$
で(R3)を満たす0は唯一


(ii)

$a\in K$に対し$b,c\in K$を両方(R4)を満たす$-a$であると仮定する.

$a+b = 0$より(R3)と合わせ
$c+(a+b)=c+0=c$

また
$a+c=0$


(R1)より
$a+c=c+a$
なので
$c+a=0$


より
\begin{align*}
b &= b+0 \ (\because (R3))\\
&= 0+b \ (\because (R1))\\
&= (c+a)+b\\
&= c+(a+b) \ (\because (R2))\\
&= c \\
\end{align*}
つまり(R4)を満たす$-a$は唯一


(iii)

$a\in K$に対し(R4)より
$a+(-a)=0$


(R1)より
$(-a)+a=a+(-a)$
で
$(-a)+a=0$だ.

より(ii)から
$-(-a)=a$

(iv)

$\ast$注意

$a\in K$がある$b\in K$に対して
$b+a=b$
なら
$a=0$
だ.

なぜなら
\begin{align*}
0 &= b+(-b) \ (\because (R4))\\
&= (b+a)+(-b)\\
&= (a+b)+(-b) \ (\because (R1) より b+a=a+b)\\
&= a+(b+(-b)) \ (\because (R2))\\
&= a+0 \ (\because (R4) より b+(-b)=0)\\
&= a \ (\because (R3))
\end{align*}
以下これは暗黙の了解として使う.

$a\in K$に対し
\begin{align*}
0a+0a &=(0+0)a \ (\because (R7)) \\
&= 0a \ (\because (R3)より0+0=0)
\end{align*}
より$0a=0$

(v)

$a\in K$に対し
\begin{align*}
a+(-1)a &= a1+(-1)a \ (\because (R8)よりa=a1)\\
&= 1a+(-1)a \ (\because (R5)よりa1=1a)\\
&= (1+(-1))a \ (\because (R7))\\
&= 0a \ (\because (R4)より1+(-1)=0)\\
&= 0 \ (\because (iv))\\
\end{align*}
より(ii)から(以下(ii)も暗黙の了解として使う)$(-1)a=-a$


(vi)

\begin{align*}
(-1)(-1) &= -(-1) \ (\because (v))\\
&= 1 \ (\because (iii))\\
\end{align*}

(vii)

\begin{align*}
ab+a(-b) &= a(b+(-b)) \ (\because (R7))\\
&= a0 \ (\because (R4)よりb+(-b)=0)\\
&= 0a \ (\because (R5))\\
&= 0 \ (\because (iv))\\
\end{align*}
より
$a(-b) = -ab$
\begin{align*}
ab+(-a)b &= (a+(-a))b \ (\because (R7))\\
&= 0b \ (\because (R4)よりa+(-a)=0)\\
&= 0 \ (\because (iv))\\
\end{align*}
より
$(-a)b = -ab$

(viii)

\begin{align*}
(-a)(-b) &= -a(-b) \ (\because (vii))\\
&= -(-ab) \ (\because (vii)よりa(-b)=-ab)\\
&= ab \ (\because (iii))\\
\end{align*}

(ix)

$b\neq 0$と仮定する.$b^{-1}$が存在し$bb^{-1}=1$.


この時
\begin{align*}
a &= a1 \ (\because (R8))\\
&= a(bb^{-1}) \ (\because bb^{-1}=1)\\
&= ab(b^{-1}) \ (\because (R6))\\
&= 0b^{-1}\\
&= 0 \ (\because (iv))
\end{align*}
つまり$a=0$または$b=0$

(x)

\begin{align*}
(-a)(-(a^{-1})) &= aa^{-1} \ (\because (viii))\\
&= 1 \ (\because (R9))\\
\end{align*}
(ii)と同様に(R9)を満たす$a^{-1}$は各$a\in K,a\neq 0$に対し唯一なので(以下これは暗黙の了解として使う).
$(-a)^{-1}=-(a^{-1})$

(xi)

\begin{align*}
(ab)(b^{-1}a^{-1})&=((ab)b^{-1})a^{-1} \ (\because (R6))\\
&=(a(bb^{-1}))a^{-1} \ (\because (R6) より(ab)b^{-1}=a(bb^{-1}))\\
&=(a1)a^{-1} \ (\because (R9) よりbb^{-1}=1)\\
&=aa^{-1} \ (\because (R8) よりa1=a)\\
&=1 \ (\because (R9))\\
\end{align*}
より
$(ab)^{-1}=b^{-1}a^{-1}$


問2(i)


$\Rightarrow$


$a\leqq b$
と(R15)より
$a+(-a)\leqq b+(-a)$

より
$0\leqq b-a$


$\Leftarrow$


$0\leqq b-a$
と(R15)より
$0+a\leqq (b-a)+a$
より
$a\leqq b$


(ii)


(i)より
$a\leqq b \Leftrightarrow 0\leqq b-a$

さらに(i)より
$-b\leqq -a \Leftrightarrow 0\leqq -a-(-b)$

以上より$-a-(-b)=b-a$と合わせて
$a\leqq b \Leftrightarrow -b\leqq -a$


(iii)


(i)と$a\leqq b$より
$b-a \geqq 0$

(i)と$c\leqq 0,0-c=-c$より
$-c\geqq 0$

より(R16)から
$(b-a)(-c)\geqq 0$で
$(b-a)(-c)=ac-bc$と(i)から
$ac\geqq bc$


(iv)

$a^{-1}\leqq 0$と仮定する.

$-a^{-1}\geqq 0$で
$a\geqq 0$とあわせ(R16)から
$a(-a^{-1})\geqq 0$

より
$-1\geqq 0$
(ii)より
$0\geqq 1$
となり矛盾.

背理法から
$a^{-1}>0$


(v)

$a\leqq b$と(R15)より
$a+c \leqq b+c$

$c\leqq d$と(R15)より
$c+b \leqq d+b$

$b+c=c+b,b+d=d+b$より
$b+c \leqq b+d$

(R13)より
$a+c\leqq b+d$


(vi)


(v)より
$a+c\leqq b+d$
は言える.

$a+c\neq b+d$
を言えばいい.

$a+c = b+d$
と仮定する.

(R11)より
$b+d\leqq a+c$

また$a\leqq b$と(ii)より
$-b\leqq -a$

(v)より$-b+(b+d)=d,-a+(a+c)=c$と合わせて
$d\leqq c$

$c<d$に矛盾し背理法から$a+c\neq b+d$


以上より
$a+c < b+d$


\subsection*{$\S$2 実数列の極限}
1)(i)

$N>|a|$となる$N\in\mathbb{N}$が存在.

$n>N$の時$|a_n| = |a_{n-1}|\frac{|a|}{n},\frac{|a|}{n}<1$ で
\[|a_n| < |a_{n-1}|\]
これを繰り返し用いると$n\geqq N$で
\[|a_n| \leqq |a_N|\]
$\epsilon > 0$に対し$n \geqq max(N+1,\frac{|aa_N|}{\epsilon}+1)$とすると
\begin{align*}
|a_n| &= |a_{n-1}|\frac{|a|}{n}\\
&\leqq \frac{|aa_N|}{n} \\
&< \epsilon
\end{align*}
より
\[a_n\to0 \ (n\to \infty)\]


(ii)

$\epsilon > 0$に対し$\epsilon'=min(1,\epsilon)$とする.

$0\leqq 1-\epsilon' < 1$なので例6より
$\displaystyle \lim_{n\to \infty}(1-\epsilon')^n = 0$

より$a>0$より$N\in\mathbb{N}$が存在し
\[n\geqq N \Rightarrow (1-\epsilon')^n < a\]

より$n\geqq N$の時
$-\epsilon \leqq -\epsilon' < \sqrt[n]{a}-1$

また二項定理より$n\geqq 1$で
\[(1+\epsilon)^n = \sum_{k=0}^n \ _nC_k  \epsilon ^k > n\epsilon\]
$M > \frac{a}{\epsilon}$を満たすように$M\in \mathbb{N}$を取ると

$n\geqq M$で
\[a < n\epsilon < (1+\epsilon)^n\]
より$\sqrt[n]{a}-1 < \epsilon$

$n\geqq max(N,M)$の時$|\sqrt[n]{a}-1| < \epsilon$で
\[a_n\to1 \ (n\to \infty)\]

(iii)

$k=2,\cdots,n$で$\frac{k}{n}\leqq 1$なので辺々掛け合わせて
\[\frac{n!}{n^{n-1}}\leqq 1\]
より$0<a_n\leqq \frac{1}{n}$

また$\displaystyle\lim_{n\to \infty}\frac{1}{n}=0$なのではさみうちの原理から
\[a_n\to 0 \ (n\to \infty)\]

(iv)

二項定理より$n\geqq 2$で
\[2^n =\sum_{k=0}^n \ _nC_k > \frac{n(n-1)}{2}\]
より$0<a_n<\frac{2}{n-1}$

また$\displaystyle\lim_{n\to \infty}\frac{2}{n-1}=0$なのではさみうちの原理から
\[a_n\to 0 \ (n\to \infty)\]

(v)

$\epsilon > 0$に対し$N>\frac{1}{\epsilon ^2}$となる$N\in \mathbb{N}$が存在.

$n\geqq N$で
\[a_n=\frac{1}{\sqrt{n+1}+\sqrt{n}}<\frac{1}{\sqrt{n}}<\epsilon\]
$a_n>0$も合わせて$n\geqq N$で$|a_n|<\epsilon$なので
\[a_n\to 0 \ (n\to \infty)\]

2)

$-1\leqq \cos(n!\pi x) \leqq 1$だ.

$\cos(n!\pi x) = \pm 1$の時$(\cos(n!\pi x))^{2m} = 1$なので$\displaystyle\lim_{m\to \infty} (\cos(n!\pi x))^{2m} = 1$

$-1<\cos(n!\pi x) < 1$の時$0 \leqq (\cos(n!\pi x))^2 < 1$なので例6より$\displaystyle \lim_{m\to \infty} (\cos(n!\pi x))^{2m} = 0$

$\cos(n!\pi x) = \pm 1\Leftrightarrow n!x \in \mathbb{Z} $だ.

$x$が有理数の時$x=\frac{p}{q},q\in \mathbb{N},p\in \mathbb{Z}$とおけ$n\geqq q$の時
\[n!x = n\cdots (q+1)\cdot (q-1) \cdots  1 \cdot p \in \mathbb{Z}\]
より$n\geqq q$で$\displaystyle\lim_{m\to \infty} (\cos(n!\pi x))^{2m} = 1$

より$\displaystyle\lim_{n\to \infty}(\lim_{m\to \infty} (\cos(n!\pi x))^{2m}) = 1$

$x$が無理数の時

$n!x$が整数と仮定する.

$x=\frac{n!x}{n!}$で分母と分子が整数なので$x$が有理数となり矛盾.

より
$n!x$は整数でなく$\displaystyle\lim_{m\to \infty} (\cos(n!\pi x))^{2m} = 0$

より$\displaystyle\lim_{n\to \infty}(\lim_{m\to \infty} (\cos(n!\pi x))^{2m}) = 0$

以上より
\[f(x) = \begin{cases}
1 & xが有理数 \\
0 & xが無理数 \\
\end{cases}\]

3)

$\epsilon > 0$とする.

$\displaystyle\lim_{n\to \infty}a_n=a$なので$n\geqq N'$なら$|a_n-a|<\frac{\epsilon}{2}$となる$N'\in \mathbb{N}$が存在.

$N=max(1,N')$とする.

$n\geqq max(N,\frac{2}{\epsilon}|\sum_{k=1}^{N-1}(a_k-a)|)$の時
\[
|\frac{1}{n}\sum_{k=1}^n a_k - a| \leqq \frac{1}{n}|\sum_{k=1}^{N-1}(a_k-a)|+\frac{1}{n}\sum_{k=N}^{n}|a_k-a|<\frac{\epsilon}{2} + \frac{n-N+1}{2n}\epsilon<\epsilon
\]
より$\displaystyle\lim_{n\to \infty}\frac{a_1+a_2+\cdots+a_n}{n}=a$

4)

$a_k\neq0$なので$a_n=a_1\cdot \frac{a_2}{a_1}\cdot \frac{a_3}{a_2}\cdots \frac{a_n}{a_{n-1}}$

より$a_k>0$に注意し
$\log a_n = \log a_1 + \log \frac{a_2}{a_1}+\log \frac{a_3}{a_2}+\cdots +\log \frac{a_n}{a_{n-1}}$

$n\in\mathbb{N}$に対し$a_n > 0$なので$b_n=\log \frac{a_{n+1}}{a_n}$とおける.

\[\log \sqrt[n]{a_n}=\frac{b_1+b_2+\cdots +b_n}{n} - \frac{b_n}{n}+\frac{\log a_1}{n}\]

$\displaystyle\lim_{n\to \infty}\frac{a_{n+1}}{a_n}=a$と$\log x$が連続なので$\displaystyle\lim_{n\to \infty}b_n=\log a$

より3)より$\displaystyle\lim _{n\to \infty}\frac{b_1+b_2+\cdots +b_n}{n} = \log a$

また$n\geqq N$で$|b_n - \log a|<1$となる$N\in\mathbb{N}$が存在.

$n\geqq N$で$\frac{\log a -1}{n}\leqq \frac{b_n}{n} \leqq \frac{\log a+1}{n}$で$\displaystyle\lim_{n\to\infty}\frac{\log a -1}{n} = \lim_{n\to\infty}\frac{\log a +1}{n}=0$なのではさみうちの原理から
$\displaystyle\lim_{n\to\infty} \frac{b_n}{n} = 0$

さらに$\displaystyle\lim_{n\to\infty} \frac{\log a_1}{n} = 0$なので


\[\lim_{n\to\infty}\log \sqrt[n]{a_n} = \log a\]
$e^x$は連続なので$\displaystyle\lim_{n\to\infty}\sqrt[n]{a_n} = e^{\log a} = a$

5)

$H = A\cup \{0\} \cup \{1\} \cup \cdots \cup \{m-1\}$とする.

$H$が継承的であることを示す.

$\{0\}\subset H$なので$0\in H$

$x\in H$とする

$x = 0,\cdots , m-2$の時$\{x+1\}\subset H$なので$x+1\in H$

$x = m-1$の時イ)より$m\in A$で$A\subset H$なので$x+1=m\in H$

$x\in A$の時イ)より$x\geqq m$

$x\in A,x\geqq m$なのでロ)より$x+1\in A$で$A\subset H$なので$x+1\in H$

以上より$H$は継承的.

より$\mathbb{N}\subset H$

$n\in \mathbb{N}$で$n\geqq m$とする.

$\mathbb{N}\subset H$より$n\in H$.

また$n\geqq m$なので$n \neq 0,1,\cdots , m-1$で$n\notin  \{0\} \cup \{1\} \cup \cdots \cup \{m-1\}$

より$n\in A$で$\{n\in \mathbb{N}|n\geqq m\}\subset A$

次に$n\in A$とする.

$A\subset \mathbb{N}$なので$n\in\mathbb{N}$

イ)より$n\geqq m$

より$n\in \{n\in \mathbb{N}|n\geqq m\}$で$A\subset \{n\in \mathbb{N}|n\geqq m\}$

以上より$A=\{n\in \mathbb{N}|n\geqq m\}$

6)

$n\in \mathbb{N}$に対し$A_n=\{x\in \mathbb{R}|x+n\in\mathbb{N}\}$とする.

$A_n$が継承的であることを示す.

$n\in \mathbb{N}$なので$0+n\in \mathbb{N}$で$0\in A_n$

$x\in A_n$とする.

$x+n\in \mathbb{N}$で$\mathbb{N}$が継承的なので$x+1+n\in \mathbb{N}$

より$x+1\in A_n$

以上より$A_n$は継承的で$\mathbb{N} \subset A_n$

$m\in \mathbb{N}$なら$m\in A_n$で$m+n\in\mathbb{N}$

$n\in \mathbb{N}$に対し$B_n=\{x\in \mathbb{R}|xn\in\mathbb{N}\}$とする.

$B_n$が継承的であることを示す.

$0n = 0\in \mathbb{N}$なので$0\in B_n$

$x\in B_n$とする.

$xn,n\in \mathbb{N}$なので上の結果より$xn+n=(x+1)n\in \mathbb{N}$

より$x+1\in B_n$

以上より$B_n$は継承的で$\mathbb{N} \subset B_n$

$m\in \mathbb{N}$なら$m\in B_n$で$mn\in\mathbb{N}$

$C=\{0\}\cup \{x\in \mathbb{N}|x-1\in\mathbb{N}\}$
とする.

$C$が継承的であることを示す.

$\{0\}\subset C$より$0\in C$

$x\in C$とする.

$x\in\{0\},x\in \{x\in \mathbb{N}|x-1\in\mathbb{N}\}$いずれの場合も$x\in\mathbb{N}$

$\mathbb{N}$は継承的なので$x+1\in\mathbb{N}$

また$x+1-1=x\in \mathbb{N}$なので$x+1\in C$

以上より$C$は継承的で$\mathbb{N} \subset C$

$m\in \mathbb{N}$に対し$D_m=\{x\in \mathbb{N}|m< x または m-x\in\mathbb{N}\}$とする.

$D_m$が継承的であることを示す.

$m\in \mathbb{N}$なので$m-0\in \mathbb{N}$で$0\in D_m$

$x\in D_m$とする.

$m\leqq x$の時$m<x+1$なので$x+1\in D_m$

$m>x$の時$m-x\in \mathbb{N}\subset C$

さらに$m-x\neq 0$なので$m-x-1\in\mathbb{N}$

より$x+1\in D_m$

いずれの場合も$x+1\in D_m$で$0\in D_m$と合わせて
$D_m$は継承的で$\mathbb{N} \subset D_m$

より$n\in \mathbb{N},m\geqq n$なら$m-n\in\mathbb{N}$

7)

$\mathbb{R}_+$は継承的なので$\mathbb{N}\subset \mathbb{R}_+$で$n\in \mathbb{N}$なら$n\geqq 0$なことに注意する.

$n\in \mathbb{N}$に対して$E_n=\{x\in \mathbb{N}|x\leqq nまたは n+1\leqq x\}$
とする.

また$F=\{n\in \mathbb{N}|\mathbb{N}\subset E_n\}$
とする.

$F$が継承的であることを示したい.

まず$E_0$が継承的なことを示す.

$0\in \mathbb{N}$で$0\leqq 0$より$0\in E_0$

$x\in E_0$とする.$x\in \mathbb{N}$で$x+1\in\mathbb{N}$

また$x\geqq 0$なので$1\leqq x+1$で$x+1\in E_0$

より$E_0$は継承的で$0\in F$

次に$n\in F$を仮定して$n+1\in F$を示す.

$n\in F \subset \mathbb{N}$なので$n\geqq 0$で$0\leqq n+1$で$0\in E_{n+1}$

$x\in E_{n+1}$とする.$x \in \mathbb{N} \subset E_n$なので$x\leqq n $または$n+1\leqq x$

より$x+1\leqq n+1$または$n+2\leqq x+1$

より$x+1\in E_{n+1}$

以上より$E_{n+1}$は継承的で$\mathbb{N}\subset E_{n+1}$

より$n+1\in F$

以上より$F$は継承的で$\mathbb{N}\subset F$

より$n\in \mathbb{N}$なら$\mathbb{N}\subset \{x\in \mathbb{N}|x\leqq nまたは n+1\leqq x\}$で

$n<k<n+1$となる自然数は存在しない.
\subsection*{$\S$3 実数の連続性}
1)(i)

\[a_n=\frac{\frac{1}{6}n(n+1)(2n+1)}{n^3}=\frac{1}{6}\cdot 1\cdot (1+\frac{1}{n})\cdot (2+\frac{1}{n})\xrightarrow[n\to \infty]{} \frac{1}{6}\cdot 1\cdot (1+0)\cdot (2+0)=\frac{1}{3}\]
(ii)

$a\leqq 1$のとき$M(\in\mathbb{R})$に対し$N>\sqrt{max(0,M)}$を満たす$N(\in \mathbb{N})$が存在.

$n\geqq N$で
\[a_n = \frac{n^2}{a^n}\geqq \frac{n^2}{1^n}=n^2  \geqq M\]
より$\displaystyle\lim_{n\to\infty}a_n=+\infty$

$a>1$のとき二項定理より$n\geqq3$で
\[a^n =\sum_{k=0}^n \ _nC_k(a-1)^k\cdot 1^{n-k}>\frac{1}{6}n(n-1)(n-2)(a-1)^3\]
より
\[\frac{n^2}{a^n}<\frac{6}{(1-\frac{1}{n})(1-2\cdot\frac{1}{n})(a-1)^3}\cdot\frac{1}{n}\xrightarrow[n\to \infty]{}\frac{6}{(1-0)(1-2\cdot 0)(a-1)^3}\cdot0=0\]
$a_n>0$と合わせはさみうちの原理から$\displaystyle\lim_{n\to\infty}a_n=0$

以上より
$\displaystyle\lim_{n\to\infty}a_n=\begin{cases}
+\infty & (a\leqq 1)\\
0 & (a> 1)\\
\end{cases}$

(iii)

$n\geqq2$のとき$\sqrt[n]{n}>1$で二項定理より
\[n=\sum_{k=0}^n \ _nC_k (\sqrt[n]{n}-1)^k\cdot 1^{n-k}
>\frac{n(n-1)}{2}(\sqrt[n]{n}-1)^2\]
より
\[|\sqrt[n]{n}-1| < \sqrt{\frac{2}{n-1}}\]
$\epsilon >0$に対し$N>\frac{2}{\epsilon^2}+1$を満たす$N\in\mathbb{N}$が存在.

$n\geqq max(N,2)$で$|\sqrt[n]{n}-1| < \epsilon$

つまり$\displaystyle\lim_{n\to\infty}a_n=1$

(iv)

2)より$e>1$で$n\geqq k+1$で二項定理より
\[e^n = \sum_{l=0}^n \ _nC_l(e-1)^l\cdot 1^{n-l}> \ _nC_{k+1}(e-1)^{k+1} > \frac{(n-k)^{k+1}}{(k+1)!}\cdot (e-1)^{k+1}\]
より
\[a_n <(\frac{1}{1-\frac{k}{n}})^k\cdot \frac{(k+1)!}{(e-1)^{k+1}}\frac{1}{n-k}\xrightarrow[n\to\infty]{}(\frac{1}{1-0})^k\cdot \frac{(k+1)!}{(e-1)^{k+1}}\cdot 0 = 0\]
$a_n>0$と合わせはさみうちの原理から$\displaystyle\lim_{n\to\infty}a_n=0$

(v)

$n\geqq 2$のとき2)の$e$を用いて
\[a_n=(1-\frac{1}{n^2})^n=(\frac{\frac{n+1}{n}}{\frac{n}{n-1}})^n=\frac{(1+\frac{1}{n})^n}{((1+\frac{1}{n-1})^{n-1})^\frac{1}{1-\frac{1}{n}}}\xrightarrow[n\to\infty]{}\frac{e}{e^{1}}=1\]

(vi)

$0<c<1$のとき
\[a_n < \frac{1}{c^{-n}}=c^n\xrightarrow[n\to\infty]{}0\]
$a_n>0$と合わせはさみうちの定理から$\displaystyle \lim_{n\to \infty}a_n=0$

$c=1$のとき$a_n=\frac{1}{2}\xrightarrow[n\to\infty]{}\frac{1}{2}$

$c>1$のとき$0<\frac{1}{c}<1$で$a_n=\frac{1}{(\frac{1}{c})^{-n}+(\frac{1}{c})^{n}}$なので$0<c<1$のときの結果より$\displaystyle \lim_{n\to \infty}a_n=0$

以上より$\displaystyle \lim_{n\to \infty}a_n=\begin{cases}
\frac{1}{2} & c=1\\
0 & c\neq 1\\
\end{cases}$

(vii)

$b_n=\frac{2\cdot4\cdot 6\cdots 2n}{3\cdot5\cdot 7\cdots (2n+1)}$とする.

$a_n>0,b_n>0$なので両方下に有界.

$\frac{2(n+1)-1}{2(n+1)}<1$より$a_{n+1}<a_n$

$\frac{2(n+1)}{2(n+1)+1}<1$より$b_{n+1}<b_n$

より$a_n,b_n$は両方単調減少で収束する.

それぞれ$a,b$に収束するとすると$a\geqq0,b\geqq 0$

$n\in\mathbb{N}-\{0\}$に対し$(2n)^2>(2n)^2-1\Leftrightarrow \frac{2n-1}{2n}<\frac{2n}{2n+1}$なので

$n=1,\cdots ,k$で掛けて$a_k<b_k$より$a\leqq b$

また$a_nb_n = \frac{1}{2n+1}\xrightarrow[n\to\infty]{}0$なので$ab=0$

$0\leqq a^2\leqq ab = 0$より$a=0$つまり$\displaystyle \lim_{n\to\infty}a_n=0$

2)

二項定理より
\[a_n=\sum_{k=0}^n \ _nC_k (\frac{1}{n})^k,a_{n+1}=\sum_{k=0}^{n+1} \ _{n+1}C_k (\frac{1}{n+1})^k\]
$l=0,\cdots ,k-1$に対し$n(n+1)-nl\geqq n(n+1)-(n+1)l\Leftrightarrow \frac{n+1-l}{n+1}\geqq \frac{n-l}{n}$

なので辺々掛けて$\frac{1}{k!}$で割り$ \ _nC_k (\frac{1}{n})^k \leqq \ _{n+1}C_k (\frac{1}{n+1})^k$
より
\[a_n=\sum_{k=0}^n \ _nC_k (\frac{1}{n})^k\leqq \sum_{k=0}^{n} \ _{n+1}C_k (\frac{1}{n+1})^k < \sum_{k=0}^{n+1} \ _{n+1}C_k (\frac{1}{n+1})^k=a_{n+1}\]
より$a_n$は単調増加.

また$ \ _nC_k\leqq \frac{n^k}{k!}$なので
\[a_n=\sum_{k=0}^n \ _nC_k (\frac{1}{n})^k\leqq \sum_{k=0}^{n} \frac{1}{k!}\]

$n\geqq3$で$\displaystyle \sum_{k=0}^n\frac{1}{k!} \leqq 2.9 - \frac{1}{n!}$を数学的帰納法で示す.

i) $n=3$のとき

$(左辺)=1+1+\frac{1}{2}+\frac{1}{6}<2.67 < 2.9-\frac{1}{6}=(右辺)$で成立.

ii) $n=l(\in \mathbb{N})$で成立すると仮定する.$(l\geqq 3)$

\[\sum_{k=0}^{l+1}\frac{1}{k!} \leqq 2.9-\frac{1}{l!}+\frac{1}{(l+1)!}=2.9-\frac{l}{(l+1)!}<2.9-\frac{1}{(l+1)!}\]

より$n=l+1$も成立.

i)ii)より示された.

より$a_n\leqq 2.9 -  \frac{1}{n!} < 2.9$で$a_n$は上に有界.より$a_n$は$e$に収束するとしてよく$e \leqq 2.9 < 3$

$n\geqq 2$で$a_n\geqq a_2 = \frac{9}{4}>2$
より$e>2$            

3)

$0 < a_n\leqq a_{n+1} \leqq b_{n+1} \leqq b_n$を数学的帰納法で示す.

i) $n=0$のとき

$a_0>0$だ.また$a<b$より$a<\frac{a+b}{2}<b$で$a_0 < a_1 < b_0$

より$\sqrt{a_1} < \sqrt{b_0}$で$a_1<\sqrt{a_1b_0}=b_1<b_0$

より成立.

ii) $n=k(\in\mathbb{N})$で成立すると仮定する.

$0<a_k\leqq a_{k+1}\leqq b_{k+1} \leqq b_k$

まず$a_{k+1}>0$.また$a_{k+1}\leqq b_{k+1}$より$a_{k+1}\leqq \frac{a_{k+1}+b_{k+1}}{2}=a_{k+2}\leqq b_{k+1}$

より$\sqrt{a_{k+2}} \leqq \sqrt{b_{k+1}}$で$a_{k+2}\leqq\sqrt{a_{k+2}b_{k+1}}=b_{k+2}\leqq b_{k+1}$

より$n=k+1$も成立.

i)ii)より示された.

より区間$[a_n,b_n]$は単調減少.また
\[b_{n+1}-a_{n+1}\leqq b_n-a_{n+1}=\frac{1}{2}(b_n-a_n)\]

これを繰り返し用い$b_n-a_n\leqq \frac{1}{2^n}(b_0-a_0)$

$\displaystyle \lim_{n\to\infty}\frac{1}{2^n}(b-a)=0,b_n-a_n\geqq 0$よりはさみうちの原理から$\displaystyle \lim_{n\to\infty}b_n-a_n=0$

以上より区間縮小法より$a_n,b_n$は収束し$\displaystyle \lim_{n\to \infty}a_n=\lim_{n\to\infty}b_n$で
この値を$l$とおける.

$0<a<b$より$0<x<\frac{\pi}{2}$としてよい.

$a_n = \frac{\sin x\cos \frac{x}{2^n}}{2^n\sin \frac{x}{2^n}}b,b_n= \frac{\sin x}{2^n\sin \frac{x}{2^n}}b$を数学的帰納法で示す.

i) $n=0$のとき

\[a_0=b\cos x = \frac{\sin x \cos \frac{x}{2^0}}{2^0\sin\frac{x}{2^0}}b,
b_0=b = \frac{\sin x}{2^0\sin\frac{x}{2^0}}b \]

より成立する.

ii) $n=k(\in \mathbb{N})$で成立すると仮定する.

\[a_{k+1}=\frac{a_k+b_k}{2}=\frac{\sin x}{2^k\sin \frac{x}{2^k}}b\cdot \frac{1+\cos \frac{x}{2^k}}{2}= \frac{\sin x\cos ^2 \frac{x}{2^{k+1}}}{2^k\sin \frac{x}{2^k}}b= \frac{\sin x\cos ^2 \frac{x}{2^{k+1}}}{2^{k+1}\sin \frac{x}{2^{k+1}}\cos \frac{x}{2^{k+1}}}b=\frac{\sin x\cos \frac{x}{2^{k+1}}}{2^{k+1}\sin \frac{x}{2^{k+1}}}b\]
\[b_{k+1}=\sqrt{a_{k+1}b_k}=\sqrt{\frac{\sin x\cos \frac{x}{2^{k+1}}}{2^{k+1}\sin \frac{x}{2^{k+1}}}b\cdot \frac{\sin x}{2^k\sin \frac{x}{2^k}}b}=\sqrt{\frac{\sin x\cos \frac{x}{2^{k+1}}}{2^{k+1}\sin \frac{x}{2^{k+1}}}b\cdot \frac{\sin x}{2^{k+1}\sin \frac{x}{2^{k+1}}\cos \frac{x}{2^{k+1}}}b}= \frac{\sin x}{2^{k+1}\sin \frac{x}{2^{k+1}}}b\]
より$n=k+1$も成立.

i)ii)より示された.

$\displaystyle \lim_{n\to\infty}\frac{x}{2^n}=0$より$\displaystyle \lim_{n\to\infty}\frac{\frac{x}{2^n}}{\sin \frac{x}{2^n}}=1$で
\[b_n= \frac{\sin x}{x}\cdot\frac{\frac{x}{2^n}}{\sin \frac{x}{2^n}}b\xrightarrow[n\to\infty]{} \frac{\sin x}{x}b\]
より$l=\frac{\sin x}{x}b$

(おまけ)

$a=\frac{1}{4},b=\frac{1}{2\sqrt{2}}$のとき$x=\frac{\pi}{4}$

直径$1$の円の中心を$O$,この円に外接,内接する正$2^{n+2}$角形の辺の1つをそれぞれ$AB,A'B'$とする.

また$AB,A'B'$の中点をそれぞれ$M,M'$とする.

\[\angle AOM = \angle A'OM' = \frac{2\pi}{2\cdot 2^{n+2}}=\frac{\pi}{2^{n+2}}\]
$MO,A'O$は円の半径で$\frac{1}{2}$
\[AM=MO\tan \frac{\pi}{2^{n+2}} = \frac{1}{2}\tan\frac{\pi}{2^{n+2}},A'M' =A'O\sin \frac{\pi}{2^{n+2}} = \frac{1}{2}\sin\frac{\pi}{2^{n+2}}\]
$AB=2AM,A'B'=2A'M'$で$2^{n+2}$個合わせてそれぞれ$2^{n+2}\tan\frac{\pi}{2^{n+2}},2^{n+2}\sin\frac{\pi}{2^{n+2}}$

逆数を取るとそれぞれ
\[\frac{1}{2^{n+2}\tan\frac{\pi}{ 2^{n+2}}}=\frac{\sin \frac{\pi}{4}\cos\frac{\pi}{2^{n+2}}}{2^n\sin\frac{\pi}{2^{n+2}}}\frac{1}{2\sqrt{2}}=\frac{\sin x\cos \frac{x}{2^n}}{2^n\sin\frac{x}{2^n}}b=a_n\]
\[\frac{1}{2^{n+2}\sin\frac{\pi}{ 2^{n+2}}}=\frac{\sin \frac{\pi}{4}}{2^n\sin\frac{\pi}{2^{n+2}}}\frac{1}{2\sqrt{2}}=\frac{\sin x}{2^n\sin\frac{x}{2^n}}b=b_n\]
また$l=\frac{\frac{1}{\sqrt{2}}}{\frac{\pi}{4}}\cdot \frac{1}{2\sqrt{2}}=\frac{1}{\pi}$

9)

$k_0+\frac{1}{k_1+\frac{1}{k_2+\cdots \frac{1}{k_n}}}$を$[k_0;k_1,k_2,\cdots,k_n]$と書く事にする.

$k_n>0 \ (n > 0)$を満たす整数列$(k_n)_{n\in\mathbb{N}}$に対し$(p'_n)_{n\in\mathbb{N}-\{0\}},(q'_n)_{n\in\mathbb{N}-\{0\}}$を以下のように定義する.
\[p'_n=\begin{cases}
k_0 & n=1 \\
k_0k_1+1 & n=2 \\
p'_{n-1}k_{n-1}+p'_{n-2} & n\geqq 3
\end{cases}, \ q'_n=\begin{cases}
1 & n=1 \\
k_1 & n=2 \\
q'_{n-1}k_{n-1}+q'_{n-2} & n\geqq 3
\end{cases}\]
$n\in \mathbb{N}-\{0\},t>1$に対し$[k_0;\cdots ,k_n,t]=\frac{p'_{n+1}t+p'_n}{q'_{n+1}t+q'_n}$を数学的帰納法で示す.

i) $n=1$の時

\begin{align*}
[k_0;k_1,t]&=k_0+\frac{1}{k_1+\frac{1}{t}}\\
&=k_0+\frac{t}{tk_1+1}\\
&=\frac{(k_0k_1+1)t+k_0}{k_1t+1}\\
&=\frac{p'_2t+p'_1}{q'_2t+q'_1}
\end{align*}

ii) $n=m(\in \mathbb{N}-\{0\})$の時に成立すると仮定する.

\begin{align*}
[k_0;\cdots ,k_{m+1},t]&=[k_0;\cdots ,k_{m},k_{m+1}+\frac{1}{t}]\\
&=\frac{p'_{m+1}(k_{m+1}+\frac{1}{t})+p'_m}{q'_{m+1}(k_{m+1}+\frac{1}{t})+q'_m} \ (\because 帰納法の仮定)\\
&=\frac{(p'_{m+1}k_{m+1}+p'_m)t+p'_{m+1}}{(q'_{m+1}k_{m+1}+q'_m)t+q'_{m+1}}\\
&=\frac{p'_{m+2}t+p'_{m+1}}{q'_{m+2}t+q'_{m+1}}
\end{align*}

i)ii)より示された.

特に$t=k_{n+1}$として$[k_0;\cdots ,k_{n+1}]=\frac{p'_{n+1}k_{n+1}+p'_n}{q'_{n+1}k_{n+1}+q'_n}=\frac{p'_{n+2}}{q'_{n+2}}$

$[k_0;]=\frac{p'_1}{q'_1},[k_0;k_1]=\frac{p'_2}{q'_2}$と合わせて

$[k_0;\cdots ,k_n]=\frac{p'_{n+1}}{q'_{n+1}}$が任意の自然数で成り立つ.

次に$p'_{n+2}q'_{n+1}-p'_{n+1}q'_{n+2}=(-1)^n$を数学的帰納法で示す.

i) $n=0$の時

$p'_2q'_1-p'_1q'_2=(k_0k_1+1)\cdot 1-k_0k_1=1=(-1)^0$で成立.

ii) $n=m(\in \mathbb{N})$の時に成立すると仮定する.
\begin{align*}
p'_{m+3}q'_{m+2}-p'_{m+2}q'_{m+3}&=(p'_{m+2}k_{m+2}+p'_{m+1})q'_{m+2}-p'_{m+2}(q'_{m+2}k_{m+2}+q'_{m+1})\\
&=-(p'_{m+2}q'_{m+1}-p'_{m+1}q'_{m+2})\\
&=(-1)^{m+1}
\end{align*}
で$n=m+1$も成立.

i)ii)より示された.

より$p'_n,q'_n$の公約数は1の約数で1.つまり$p'_n,q'_n$は互いに素.

$x\in \mathbb{R}-\mathbb{Q}$に対し問題文のように変数を設定すると$x$の連分数展開は$k_0=[x]$として$[k_0;\cdots,k_n,\cdots]$

上で示したことより$a_n=\frac{q'_n}{p'_n}$

$p'_n,q'_n$は互いに素なので$p_n=p'_n,q_n=q'_n$

より$x=\frac{p_nx_n+p_{n-1}}{q_nx_n+q_{n-1}}$

$q_n \geqq n-1$を数学的帰納法で示す.

i) $n=1,2,3$の時

$q_1=1\geqq 0$

$k_1$は正整数なので$q_2=k_1\geqq 1$

$k_2$は正整数なので$q_3=q_2k_2+q_1\geqq 1\cdot 1 + 1 \geqq 2$

ii) $n=m,m+1,m+2 (m\in \mathbb{N}-\{0\})$の時に成立すると仮定する.

$k_{m+2}$は正整数なので$q_{m+3}=q_{m+2}k_{m+2}+q_{m+1} \geqq (m+1)\cdot 1 +m = 2m+1 \geqq m+2 $

で$n=m+3$も成立.

i)ii)より示された.

$n > 1$で
\begin{align*}
|x-a_n| &= |\frac{p_nx_n+p_{n-1}}{q_nx_n+q_{n-1}}-\frac{p_n}{q_n}|\\
&=|\frac{q_np_{n-1}-p_nq_{n-1}}{q_n(q_nx_n+q_{n-1})}|\\
&= |\frac{1}{q_n(q_nx_n+q_{n-1})}| \ (\because p_nq_{n-1}-p_{n-1}q_n=(-1)^{n-1})\\
&\leqq |\frac{1}{q_n^2}| \ (\because x_n >1 ,q_{n-1}\geqq n-2\geqq 0) 
\end{align*}
$n=0$も
\[|x-a_1|=x-[x]\leq 1=\frac{1}{q_1^2}\]
$\epsilon(>0)$に対し$N\in\mathbb{N}$が$N>\frac{1}{\sqrt{\epsilon}}+1$を満たすとする.

$n\geqq N$なら$q_n > \frac{1}{\sqrt{\epsilon}}$で$|x-a_n|\leqq \frac{1}{q_1^2}<\epsilon$

つまり$\displaystyle\lim_{n\to \infty}a_n=x$

(捕捉)

上の答案は無理数の連分数展開が無限に続くことを既知としている.その証明をここに書く.

$[k_0;k_1,k_2,\cdots,k_n]$が有理数なことを数学的帰納法で示す.ただし$k_i$は$i=0,\cdots, n$で整数で$k_i>0 \ (i>0)$

i) $n=0$のとき

$k_0\in\mathbb{Z}\subset \mathbb{Q}$
で成立.

ii) $n=m(\in\mathbb{N})$で成立すると仮定する.
\[[k_0;k_1,k_2,\cdots ,k_{m+1}]=k_0+\frac{1}{[k_1;k_2,\cdots ,k_{m+1}]}\]
帰納法の仮定より$[k_1;k_2,\cdots ,k_{m+1}]=\frac{q}{p} \ (p,q\in\mathbb{Z})$とおけ$[k_0;k_1,k_2,\cdots ,k_{m+1}]=\frac{k_0q+p}{q}$でこれは有理数.

より$n=m+1$のときも成立.

i)ii)より示された.

つまり$x$の連分数展開が有限なら$x$は有理数.

対偶を取り無理数の連分数展開は無限.

(おまけ)

有理数の連分数展開が有限の証明

$p\in\mathbb{Z},q\in \mathbb{N}-\{0\}$に対し$\frac{p}{q}$の連分数展開が有限であることを$q$に関する数学的帰納法で示す.

i) $q=1$のとき

$\frac{p}{q}=p$で連分数展開は$[p;]$より成立.

ii) $q=1,\cdots,m$で成立すると仮定する.

$p$が$m+1$の倍数のとき$\frac{p}{m+1}=[\frac{p}{m+1};]$

そうでないとき$p=(m+1)a+b$とおける. \ $(a\in\mathbb{Z} ,b=1,\cdots,m)$

\[\frac{p}{m+1}=a+\frac{1}{\frac{m+1}{b}}\]
帰納法の仮定から$\frac{m+1}{b}=[k_1;k_2,\cdots,k_n]$とおける.

$\frac{p}{m+1}=[a;k_1,k_2,\cdots,k_n]$で連分数展開は有限.

より$q=m+1$も成立.

i)ii)より有理数の連分数展開が有限
\end{document}
