\documentclass{jsarticle}
\usepackage{amsmath,amssymb}
\usepackage{bm}
\usepackage[dvipdfmx]{graphicx}
\usepackage{listings,jlisting}
\usepackage{siunitx}
\usepackage{theorem}
\title{解析入門 解答}
\author{河村遼}
\begin{document}
\maketitle{}
\section*{第I章実数と連続}
\subsection*{$\S$1 実数}
問1(i)

$a,b\in K$が両方(R3)を満たす0であると仮定する.

$a$が(R3)を満たす$0$なので
$b+a=b$

$b$も(R3)を満たす$0$なので
$a+b=a$

また(R1)より
$a+b=b+a$


以上より
$a=b$
で(R3)を満たす0は唯一


(ii)

$a\in K$に対し$b,c\in K$を両方(R4)を満たす$-a$であると仮定する.

$a+b = 0$より(R3)と合わせ
$c+(a+b)=c+0=c$

また
$a+c=0$


(R1)より
$a+c=c+a$
なので
$c+a=0$


より
\begin{align*}
b &= b+0 \ (\because (R3))\\
&= 0+b \ (\because (R1))\\
&= (c+a)+b\\
&= c+(a+b) \ (\because (R2))\\
&= c \\
\end{align*}
つまり(R4)を満たす$-a$は唯一


(iii)

$a\in K$に対し(R4)より
$a+(-a)=0$


(R1)より
$(-a)+a=a+(-a)$
で
$(-a)+a=0$だ.

より(ii)から
$-(-a)=a$

(iv)

$\ast$注意

$a\in K$がある$b\in K$に対して
$b+a=b$
なら
$a=0$
だ.

なぜなら
\begin{align*}
0 &= b+(-b) \ (\because (R4))\\
&= (b+a)+(-b)\\
&= (a+b)+(-b) \ (\because (R1) より b+a=a+b)\\
&= a+(b+(-b)) \ (\because (R2))\\
&= a+0 \ (\because (R4) より b+(-b)=0)\\
&= a \ (\because (R3))
\end{align*}
以下これは暗黙の了解として使う.

$a\in K$に対し
\begin{align*}
0a+0a &=(0+0)a \ (\because (R7)) \\
&= 0a \ (\because (R3)より0+0=0)
\end{align*}
より$0a=0$

(v)

$a\in K$に対し
\begin{align*}
a+(-1)a &= a1+(-1)a \ (\because (R8)よりa=a1)\\
&= 1a+(-1)a \ (\because (R5)よりa1=1a)\\
&= (1+(-1))a \ (\because (R7))\\
&= 0a \ (\because (R7))\\
&= 0 \ (\because (iv))\\
\end{align*}
より(ii)から(以下(ii)も暗黙の了解として使う)$(-1)a=-a$


(vi)

\begin{align*}
(-1)(-1) &= -(-1) \ (\because (v))\\
&= 1 \ (\because (iii))\\
\end{align*}

(vii)

\begin{align*}
ab+a(-b) &= a(b+(-b)) \ (\because (R7))\\
&= a0 \ (\because (R4)よりb+(-b)=0)\\
&= 0a \ (\because (R5))\\
&= 0 \ (\because (iv))\\
\end{align*}
より
$a(-b) = -ab$
\begin{align*}
ab+(-a)b &= (a+(-a))b \ (\because (R7))\\
&= 0b \ (\because (R4)よりa+(-a)=0)\\
&= 0 \ (\because (iv))\\
\end{align*}
より
$(-a)b = -ab$

(viii)

\begin{align*}
(-a)(-b) &= -a(-b) \ (\because (vii))\\
&= -(-ab) \ (\because (vii)よりa(-b)=-ab)\\
&= ab \ (\because (iii))\\
\end{align*}

(ix)

$b\neq 0$と仮定する.$b^{-1}$が存在し$bb^{-1}=1$.


この時
\begin{align*}
a &= a1 \ (\because (R8))\\
&= a(bb^{-1}) \ (\because bb^{-1}=1)\\
&= ab(b^{-1}) \ (\because (R6))\\
&= 0b^{-1}\\
&= 0 \ (\because (iv))
\end{align*}
つまり$a=0$または$b=0$

(x)

\begin{align*}
(-a)(-(a^{-1})) &= aa^{-1} \ (\because (R8))\\
&= 1 \ (\because (R8))\\
\end{align*}
(ii)と同様に(R9)を満たす$a^{-1}$は各$a\in K,a\neq 0$に対し唯一なので(以下これは暗黙の了解として使う).
$(-a)^{-1}=-(a^{-1})$

(xi)

\begin{align*}
(ab)(b^{-1}a^{-1})&=((ab)b^{-1})a^{-1} \ (\because (R6))\\
&=(a(bb^{-1}))a^{-1} \ (\because (R6) より(ab)b^{-1}=a(bb^{-1}))\\
&=(a1)a^{-1} \ (\because (R9) よりbb^{-1}=1)\\
&=aa^{-1} \ (\because (R8) よりa1=a)\\
&=1 \ (\because (R9))\\
\end{align*}
より
$(ab)^{-1}=b^{-1}a^{-1}$


問2(i)

$\Rightarrow$


$a\leqq b$
と(R15)より
$a+(-a)\leqq b+(-a)$

より
$0\leqq b-a$


$\Leftarrow$


$0\leqq b-a$
と(R15)より
$0+a\leqq (b-a)+a$
より
$a\leqq b$


(ii)


(i)より
$a\leqq b \Leftrightarrow 0\leqq b-a$

さらに(i)より
$-b\leqq -a \Leftrightarrow 0\leqq -a-(-b)$

以上より$-a-(-b)=b-a$と合わせて
$a\leqq b \Leftrightarrow -b\leqq -a$


(iii)


(i)と$a\leqq b$より
$b-a \geqq 0$

(i)と$c\leqq 0,-0=0$より
$-c\geqq 0$

より(R16)から
$(b-a)(-c)\geqq 0$で
$(b-a)(-c)=ac-bc$と(i)から
$ac\geqq bc$


(iv)

$a^{-1}\leqq 0$と仮定する.

$-a^{-1}\geqq 0$で
$a\geqq 0$とあわせ(R16)から
$a(-a^{-1})\geqq 0$

より
$-1\geqq 0$
(ii)より
$0\geqq 1$
となり矛盾.

背理法から
$a^{-1}>0$


(v)

$a\leqq b$と(R15)より
$a+c \leqq b+c$

$c\leqq d$と(R15)より
$c+b \leqq d+b$

$b+c=c+b,b+d=d+b$より
$b+c \leqq b+d$

(R13)より
$a+c\leqq b+d$


(vi)


(v)より
$a+c\leqq b+d$
は言える.

$a+c\neq b+d$
を言えばいい.

$a+c = b+d$
と仮定する.

(R11)より
$b+d\leqq a+c$

また$a\leqq b$と(ii)より
$-b\leqq -a$

(v)より$-b+(b+d)=d,-a+(a+c)=c$と合わせて
$d\leqq c$

$c<d$に矛盾し背理法から$a+c\neq b+d$


以上より
$a+c < b+d$




\end{document}
