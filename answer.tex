\documentclass{jsarticle}
\usepackage{amsmath,amssymb}
\usepackage{bm}
\usepackage[dvipdfmx]{graphicx}
\usepackage{listings,jlisting}
\usepackage{siunitx}
\usepackage{theorem}
\title{解析入門 解答}
\author{itleigns}
\begin{document}
\maketitle{}
\section*{第I章実数と連続}
\subsection*{$\S$1 実数}
問1(i)

$a,b\in K$が両方(R3)を満たす0であると仮定する.

$a$が(R3)を満たす$0$なので
$b+a=b$

$b$も(R3)を満たす$0$なので
$a+b=a$

また(R1)より
$a+b=b+a$


以上より
$a=b$
で(R3)を満たす0は唯一


(ii)

$a\in K$に対し$b,c\in K$を両方(R4)を満たす$-a$であると仮定する.

$a+b = 0$より(R3)と合わせ
$c+(a+b)=c+0=c$

また
$a+c=0$


(R1)より
$a+c=c+a$
なので
$c+a=0$


より
\begin{align*}
b &= b+0 \ (\because (R3))\\
&= 0+b \ (\because (R1))\\
&= (c+a)+b\\
&= c+(a+b) \ (\because (R2))\\
&= c \\
\end{align*}
つまり(R4)を満たす$-a$は唯一


(iii)

$a\in K$に対し(R4)より
$a+(-a)=0$


(R1)より
$(-a)+a=a+(-a)$
で
$(-a)+a=0$だ.

より(ii)から
$-(-a)=a$

(iv)

$\ast$注意

$a\in K$がある$b\in K$に対して
$b+a=b$
なら
$a=0$
だ.

なぜなら
\begin{align*}
0 &= b+(-b) \ (\because (R4))\\
&= (b+a)+(-b)\\
&= (a+b)+(-b) \ (\because (R1) より b+a=a+b)\\
&= a+(b+(-b)) \ (\because (R2))\\
&= a+0 \ (\because (R4) より b+(-b)=0)\\
&= a \ (\because (R3))
\end{align*}
以下これは暗黙の了解として使う.

$a\in K$に対し
\begin{align*}
0a+0a &=(0+0)a \ (\because (R7)) \\
&= 0a \ (\because (R3)より0+0=0)
\end{align*}
より$0a=0$

(v)

$a\in K$に対し
\begin{align*}
a+(-1)a &= a1+(-1)a \ (\because (R8)よりa=a1)\\
&= 1a+(-1)a \ (\because (R5)よりa1=1a)\\
&= (1+(-1))a \ (\because (R7))\\
&= 0a \ (\because (R4)より1+(-1)=0)\\
&= 0 \ (\because (iv))\\
\end{align*}
より(ii)から(以下(ii)も暗黙の了解として使う)$(-1)a=-a$


(vi)

\begin{align*}
(-1)(-1) &= -(-1) \ (\because (v))\\
&= 1 \ (\because (iii))\\
\end{align*}

(vii)

\begin{align*}
ab+a(-b) &= a(b+(-b)) \ (\because (R7))\\
&= a0 \ (\because (R4)よりb+(-b)=0)\\
&= 0a \ (\because (R5))\\
&= 0 \ (\because (iv))\\
\end{align*}
より
$a(-b) = -ab$
\begin{align*}
ab+(-a)b &= (a+(-a))b \ (\because (R7))\\
&= 0b \ (\because (R4)よりa+(-a)=0)\\
&= 0 \ (\because (iv))\\
\end{align*}
より
$(-a)b = -ab$

(viii)

\begin{align*}
(-a)(-b) &= -a(-b) \ (\because (vii))\\
&= -(-ab) \ (\because (vii)よりa(-b)=-ab)\\
&= ab \ (\because (iii))\\
\end{align*}

(ix)

$b\neq 0$と仮定する.$b^{-1}$が存在し$bb^{-1}=1$.


このとき
\begin{align*}
a &= a1 \ (\because (R8))\\
&= a(bb^{-1}) \ (\because bb^{-1}=1)\\
&= ab(b^{-1}) \ (\because (R6))\\
&= 0b^{-1}\\
&= 0 \ (\because (iv))
\end{align*}
つまり$a=0$または$b=0$

(x)

\begin{align*}
(-a)(-(a^{-1})) &= aa^{-1} \ (\because (viii))\\
&= 1 \ (\because (R9))\\
\end{align*}
(ii)と同様に(R9)を満たす$a^{-1}$は各$a\in K,a\neq 0$に対し唯一なので(以下これは暗黙の了解として使う).
$(-a)^{-1}=-(a^{-1})$

(xi)

\begin{align*}
(ab)(b^{-1}a^{-1})&=((ab)b^{-1})a^{-1} \ (\because (R6))\\
&=(a(bb^{-1}))a^{-1} \ (\because (R6) より(ab)b^{-1}=a(bb^{-1}))\\
&=(a1)a^{-1} \ (\because (R9) よりbb^{-1}=1)\\
&=aa^{-1} \ (\because (R8) よりa1=a)\\
&=1 \ (\because (R9))\\
\end{align*}
より
$(ab)^{-1}=b^{-1}a^{-1}$


問2(i)


$\Rightarrow$


$a\leqq b$
と(R15)より
$a+(-a)\leqq b+(-a)$

より
$0\leqq b-a$


$\Leftarrow$


$0\leqq b-a$
と(R15)より
$0+a\leqq (b-a)+a$
より
$a\leqq b$


(ii)


(i)より
$a\leqq b \Leftrightarrow 0\leqq b-a$

さらに(i)より
$-b\leqq -a \Leftrightarrow 0\leqq -a-(-b)$

以上より$-a-(-b)=b-a$と合わせて
$a\leqq b \Leftrightarrow -b\leqq -a$


(iii)


(i)と$a\leqq b$より
$b-a \geqq 0$

(i)と$c\leqq 0,0-c=-c$より
$-c\geqq 0$

より(R16)から
$(b-a)(-c)\geqq 0$で
$(b-a)(-c)=ac-bc$と(i)から
$ac\geqq bc$


(iv)

$a^{-1}\leqq 0$と仮定する.

$-a^{-1}\geqq 0$で
$a\geqq 0$とあわせ(R16)から
$a(-a^{-1})\geqq 0$

より
$-1\geqq 0$
(ii)より
$0\geqq 1$
となり矛盾.

背理法から
$a^{-1}>0$


(v)

$a\leqq b$と(R15)より
$a+c \leqq b+c$

$c\leqq d$と(R15)より
$c+b \leqq d+b$

$b+c=c+b,b+d=d+b$より
$b+c \leqq b+d$

(R13)より
$a+c\leqq b+d$


(vi)


(v)より
$a+c\leqq b+d$
は言える.

$a+c\neq b+d$
を言えばいい.

$a+c = b+d$
と仮定する.

(R11)より
$b+d\leqq a+c$

また$a\leqq b$と(ii)より
$-b\leqq -a$

(v)より$-b+(b+d)=d,-a+(a+c)=c$と合わせて
$d\leqq c$

$c<d$に矛盾し背理法から$a+c\neq b+d$


以上より
$a+c < b+d$


\subsection*{$\S$2 実数列の極限}
1)(i)

$N>|a|$となる$N\in\mathbb{N}$が存在.

$n>N$のとき$|a_n| = |a_{n-1}|\frac{|a|}{n},\frac{|a|}{n}<1$ で
\[|a_n| < |a_{n-1}|\]
これを繰り返し用いると$n\geqq N$で
\[|a_n| \leqq |a_N|\]
$\epsilon > 0$に対し$n \geqq max(N+1,\frac{|aa_N|}{\epsilon}+1)$とすると
\begin{align*}
|a_n| &= |a_{n-1}|\frac{|a|}{n}\\
&\leqq \frac{|aa_N|}{n} \\
&< \epsilon
\end{align*}
より
\[a_n\to0 \ (n\to \infty)\]


(ii)

$\epsilon > 0$に対し$\epsilon'=min(1,\epsilon)$とする.

$0\leqq 1-\epsilon' < 1$なので例6より
$\displaystyle \lim_{n\to \infty}(1-\epsilon')^n = 0$

より$a>0$より$N\in\mathbb{N}$が存在し
\[n\geqq N \Rightarrow (1-\epsilon')^n < a\]

より$n\geqq N$のとき
$-\epsilon \leqq -\epsilon' < \sqrt[n]{a}-1$

また二項定理より$n\geqq 1$で
\[(1+\epsilon)^n = \sum_{k=0}^n \ _nC_k  \epsilon ^k > n\epsilon\]
$M > \frac{a}{\epsilon}$を満たすように$M\in \mathbb{N}$を取ると

$n\geqq M$で
\[a < n\epsilon < (1+\epsilon)^n\]
より$\sqrt[n]{a}-1 < \epsilon$

$n\geqq max(N,M)$のとき$|\sqrt[n]{a}-1| < \epsilon$で
\[a_n\to1 \ (n\to \infty)\]

(iii)

$k=2,\cdots,n$で$\frac{k}{n}\leqq 1$なので辺々掛け合わせて
\[\frac{n!}{n^{n-1}}\leqq 1\]
より$0<a_n\leqq \frac{1}{n}$

また$\displaystyle\lim_{n\to \infty}\frac{1}{n}=0$なのではさみうちの原理から
\[a_n\to 0 \ (n\to \infty)\]

(iv)

二項定理より$n\geqq 2$で
\[2^n =\sum_{k=0}^n \ _nC_k > \frac{n(n-1)}{2}\]
より$0<a_n<\frac{2}{n-1}$

また$\displaystyle\lim_{n\to \infty}\frac{2}{n-1}=0$なのではさみうちの原理から
\[a_n\to 0 \ (n\to \infty)\]

(v)

$\epsilon > 0$に対し$N>\frac{1}{\epsilon ^2}$となる$N\in \mathbb{N}$が存在.

$n\geqq N$で
\[a_n=\frac{1}{\sqrt{n+1}+\sqrt{n}}<\frac{1}{\sqrt{n}}<\epsilon\]
$a_n>0$も合わせて$n\geqq N$で$|a_n|<\epsilon$なので
\[a_n\to 0 \ (n\to \infty)\]

2)

$-1\leqq \cos(n!\pi x) \leqq 1$だ.

$\cos(n!\pi x) = \pm 1$のとき$(\cos(n!\pi x))^{2m} = 1$なので$\displaystyle\lim_{m\to \infty} (\cos(n!\pi x))^{2m} = 1$

$-1<\cos(n!\pi x) < 1$のとき$0 \leqq (\cos(n!\pi x))^2 < 1$なので例6より$\displaystyle \lim_{m\to \infty} (\cos(n!\pi x))^{2m} = 0$

$\cos(n!\pi x) = \pm 1\Leftrightarrow n!x \in \mathbb{Z} $だ.

$x$が有理数のとき$x=\frac{p}{q},q\in \mathbb{N},p\in \mathbb{Z}$とおけ$n\geqq q$のとき
\[n!x = n\cdots (q+1)\cdot (q-1) \cdots  1 \cdot p \in \mathbb{Z}\]
より$n\geqq q$で$\displaystyle\lim_{m\to \infty} (\cos(n!\pi x))^{2m} = 1$

より$\displaystyle\lim_{n\to \infty}(\lim_{m\to \infty} (\cos(n!\pi x))^{2m}) = 1$

$x$が無理数のとき

$n!x$が整数と仮定する.

$x=\frac{n!x}{n!}$で分母と分子が整数なので$x$が有理数となり矛盾.

より
$n!x$は整数でなく$\displaystyle\lim_{m\to \infty} (\cos(n!\pi x))^{2m} = 0$

より$\displaystyle\lim_{n\to \infty}(\lim_{m\to \infty} (\cos(n!\pi x))^{2m}) = 0$

以上より
\[f(x) = \begin{cases}
1 & xが有理数 \\
0 & xが無理数 \\
\end{cases}\]

3)

$\epsilon > 0$とする.

$\displaystyle\lim_{n\to \infty}a_n=a$なので$n\geqq N'$なら$|a_n-a|<\frac{\epsilon}{2}$となる$N'\in \mathbb{N}$が存在.

$N=max(1,N')$とする.

$n\geqq max(N,\frac{2}{\epsilon}|\sum_{k=1}^{N-1}(a_k-a)|)$のとき
\[
|\frac{1}{n}\sum_{k=1}^n a_k - a| \leqq \frac{1}{n}|\sum_{k=1}^{N-1}(a_k-a)|+\frac{1}{n}\sum_{k=N}^{n}|a_k-a|<\frac{\epsilon}{2} + \frac{n-N+1}{2n}\epsilon<\epsilon
\]
より$\displaystyle\lim_{n\to \infty}\frac{a_1+a_2+\cdots+a_n}{n}=a$

4)

$a_k\neq0$なので$a_n=a_1\cdot \frac{a_2}{a_1}\cdot \frac{a_3}{a_2}\cdots \frac{a_n}{a_{n-1}}$

より$a_k>0$に注意し
$\log a_n = \log a_1 + \log \frac{a_2}{a_1}+\log \frac{a_3}{a_2}+\cdots +\log \frac{a_n}{a_{n-1}}$

$n\in\mathbb{N}$に対し$a_n > 0$なので$b_n=\log \frac{a_{n+1}}{a_n}$とおける.

\[\log \sqrt[n]{a_n}=\frac{b_1+b_2+\cdots +b_n}{n} - \frac{b_n}{n}+\frac{\log a_1}{n}\]

$\displaystyle\lim_{n\to \infty}\frac{a_{n+1}}{a_n}=a$と$\log x$が連続なので$\displaystyle\lim_{n\to \infty}b_n=\log a$

より3)より$\displaystyle\lim _{n\to \infty}\frac{b_1+b_2+\cdots +b_n}{n} = \log a$

また$n\geqq N$で$|b_n - \log a|<1$となる$N\in\mathbb{N}$が存在.

$n\geqq N$で$\frac{\log a -1}{n}\leqq \frac{b_n}{n} \leqq \frac{\log a+1}{n}$で$\displaystyle\lim_{n\to\infty}\frac{\log a -1}{n} = \lim_{n\to\infty}\frac{\log a +1}{n}=0$なのではさみうちの原理から
$\displaystyle\lim_{n\to\infty} \frac{b_n}{n} = 0$

さらに$\displaystyle\lim_{n\to\infty} \frac{\log a_1}{n} = 0$なので


\[\lim_{n\to\infty}\log \sqrt[n]{a_n} = \log a\]
$e^x$は連続なので$\displaystyle\lim_{n\to\infty}\sqrt[n]{a_n} = e^{\log a} = a$

5)

$H = A\cup \{0\} \cup \{1\} \cup \cdots \cup \{m-1\}$とする.

$H$が継承的であることを示す.

$\{0\}\subset H$なので$0\in H$

$x\in H$とする

$x = 0,\cdots , m-2$のとき$\{x+1\}\subset H$なので$x+1\in H$

$x = m-1$のときイ)より$m\in A$で$A\subset H$なので$x+1=m\in H$

$x\in A$のときイ)より$x\geqq m$

$x\in A,x\geqq m$なのでロ)より$x+1\in A$で$A\subset H$なので$x+1\in H$

以上より$H$は継承的.

より$\mathbb{N}\subset H$

$n\in \mathbb{N}$で$n\geqq m$とする.

$\mathbb{N}\subset H$より$n\in H$.

また$n\geqq m$なので$n \neq 0,1,\cdots , m-1$で$n\notin  \{0\} \cup \{1\} \cup \cdots \cup \{m-1\}$

より$n\in A$で$\{n\in \mathbb{N}|n\geqq m\}\subset A$

次に$n\in A$とする.

$A\subset \mathbb{N}$なので$n\in\mathbb{N}$

イ)より$n\geqq m$

より$n\in \{n\in \mathbb{N}|n\geqq m\}$で$A\subset \{n\in \mathbb{N}|n\geqq m\}$

以上より$A=\{n\in \mathbb{N}|n\geqq m\}$

6)

$n\in \mathbb{N}$に対し$A_n=\{x\in \mathbb{R}|x+n\in\mathbb{N}\}$とする.

$A_n$が継承的であることを示す.

$n\in \mathbb{N}$なので$0+n\in \mathbb{N}$で$0\in A_n$

$x\in A_n$とする.

$x+n\in \mathbb{N}$で$\mathbb{N}$が継承的なので$x+1+n\in \mathbb{N}$

より$x+1\in A_n$

以上より$A_n$は継承的で$\mathbb{N} \subset A_n$

$m\in \mathbb{N}$なら$m\in A_n$で$m+n\in\mathbb{N}$

$n\in \mathbb{N}$に対し$B_n=\{x\in \mathbb{R}|xn\in\mathbb{N}\}$とする.

$B_n$が継承的であることを示す.

$0n = 0\in \mathbb{N}$なので$0\in B_n$

$x\in B_n$とする.

$xn,n\in \mathbb{N}$なので上の結果より$xn+n=(x+1)n\in \mathbb{N}$

より$x+1\in B_n$

以上より$B_n$は継承的で$\mathbb{N} \subset B_n$

$m\in \mathbb{N}$なら$m\in B_n$で$mn\in\mathbb{N}$

$C=\{0\}\cup \{x\in \mathbb{N}|x-1\in\mathbb{N}\}$
とする.

$C$が継承的であることを示す.

$\{0\}\subset C$より$0\in C$

$x\in C$とする.

$x\in\{0\},x\in \{x\in \mathbb{N}|x-1\in\mathbb{N}\}$いずれの場合も$x\in\mathbb{N}$

$\mathbb{N}$は継承的なので$x+1\in\mathbb{N}$

また$x+1-1=x\in \mathbb{N}$なので$x+1\in C$

以上より$C$は継承的で$\mathbb{N} \subset C$

$m\in \mathbb{N}$に対し$D_m=\{x\in \mathbb{N}|m< x または m-x\in\mathbb{N}\}$とする.

$D_m$が継承的であることを示す.

$m\in \mathbb{N}$なので$m-0\in \mathbb{N}$で$0\in D_m$

$x\in D_m$とする.

$m\leqq x$のとき$m<x+1$なので$x+1\in D_m$

$m>x$のとき$m-x\in \mathbb{N}\subset C$

さらに$m-x\neq 0$なので$m-x-1\in\mathbb{N}$

より$x+1\in D_m$

いずれの場合も$x+1\in D_m$で$0\in D_m$と合わせて
$D_m$は継承的で$\mathbb{N} \subset D_m$

より$n\in \mathbb{N},m\geqq n$なら$m-n\in\mathbb{N}$

7)

$\mathbb{R}_+$は継承的なので$\mathbb{N}\subset \mathbb{R}_+$で$n\in \mathbb{N}$なら$n\geqq 0$なことに注意する.

$n\in \mathbb{N}$に対して$E_n=\{x\in \mathbb{N}|x\leqq nまたは n+1\leqq x\}$
とする.

また$F=\{n\in \mathbb{N}|\mathbb{N}\subset E_n\}$
とする.

$F$が継承的であることを示したい.

まず$E_0$が継承的なことを示す.

$0\in \mathbb{N}$で$0\leqq 0$より$0\in E_0$

$x\in E_0$とする.$x\in \mathbb{N}$で$x+1\in\mathbb{N}$

また$x\geqq 0$なので$1\leqq x+1$で$x+1\in E_0$

より$E_0$は継承的で$0\in F$

次に$n\in F$を仮定して$n+1\in F$を示す.

$n\in F \subset \mathbb{N}$なので$n\geqq 0$で$0\leqq n+1$で$0\in E_{n+1}$

$x\in E_{n+1}$とする.$x \in \mathbb{N} \subset E_n$なので$x\leqq n $または$n+1\leqq x$

より$x+1\leqq n+1$または$n+2\leqq x+1$

より$x+1\in E_{n+1}$

以上より$E_{n+1}$は継承的で$\mathbb{N}\subset E_{n+1}$

より$n+1\in F$

以上より$F$は継承的で$\mathbb{N}\subset F$

より$n\in \mathbb{N}$なら$\mathbb{N}\subset \{x\in \mathbb{N}|x\leqq nまたは n+1\leqq x\}$で

$n<k<n+1$となる自然数は存在しない.
\subsection*{$\S$3 実数の連続性}
1)(i)

\[a_n=\frac{\frac{1}{6}n(n+1)(2n+1)}{n^3}=\frac{1}{6}\cdot 1\cdot (1+\frac{1}{n})\cdot (2+\frac{1}{n})\xrightarrow[n\to \infty]{} \frac{1}{6}\cdot 1\cdot (1+0)\cdot (2+0)=\frac{1}{3}\]
(ii)

$a\leqq 1$のとき$M(\in\mathbb{R})$に対し$N>\sqrt{max(0,M)}$を満たす$N(\in \mathbb{N})$が存在.

$n\geqq N$で
\[a_n = \frac{n^2}{a^n}\geqq \frac{n^2}{1^n}=n^2  \geqq M\]
より$\displaystyle\lim_{n\to\infty}a_n=+\infty$

$a>1$のとき二項定理より$n\geqq3$で
\[a^n =\sum_{k=0}^n \ _nC_k(a-1)^k\cdot 1^{n-k}>\frac{1}{6}n(n-1)(n-2)(a-1)^3\]
より
\[\frac{n^2}{a^n}<\frac{6}{(1-\frac{1}{n})(1-2\cdot\frac{1}{n})(a-1)^3}\cdot\frac{1}{n}\xrightarrow[n\to \infty]{}\frac{6}{(1-0)(1-2\cdot 0)(a-1)^3}\cdot0=0\]
$a_n>0$と合わせはさみうちの原理から$\displaystyle\lim_{n\to\infty}a_n=0$

以上より
$\displaystyle\lim_{n\to\infty}a_n=\begin{cases}
+\infty & (a\leqq 1)\\
0 & (a> 1)\\
\end{cases}$

(iii)

$n\geqq2$のとき$\sqrt[n]{n}>1$で二項定理より
\[n=\sum_{k=0}^n \ _nC_k (\sqrt[n]{n}-1)^k\cdot 1^{n-k}
>\frac{n(n-1)}{2}(\sqrt[n]{n}-1)^2\]
より
\[|\sqrt[n]{n}-1| < \sqrt{\frac{2}{n-1}}\]
$\epsilon >0$に対し$N>\frac{2}{\epsilon^2}+1$を満たす$N\in\mathbb{N}$が存在.

$n\geqq max(N,2)$で$|\sqrt[n]{n}-1| < \epsilon$

つまり$\displaystyle\lim_{n\to\infty}a_n=1$

(iv)

2)より$e>1$で$n\geqq k+1$で二項定理より
\[e^n = \sum_{l=0}^n \ _nC_l(e-1)^l\cdot 1^{n-l}> \ _nC_{k+1}(e-1)^{k+1} > \frac{(n-k)^{k+1}}{(k+1)!}\cdot (e-1)^{k+1}\]
より
\[a_n <(\frac{1}{1-\frac{k}{n}})^k\cdot \frac{(k+1)!}{(e-1)^{k+1}}\frac{1}{n-k}\xrightarrow[n\to\infty]{}(\frac{1}{1-0})^k\cdot \frac{(k+1)!}{(e-1)^{k+1}}\cdot 0 = 0\]
$a_n>0$と合わせはさみうちの原理から$\displaystyle\lim_{n\to\infty}a_n=0$

(v)

$n\geqq 2$のとき2)の$e$を用いて
\[a_n=(1-\frac{1}{n^2})^n=(\frac{\frac{n+1}{n}}{\frac{n}{n-1}})^n=\frac{(1+\frac{1}{n})^n}{((1+\frac{1}{n-1})^{n-1})^\frac{1}{1-\frac{1}{n}}}\xrightarrow[n\to\infty]{}\frac{e}{e^{1}}=1\]

(vi)

$0<c<1$のとき
\[a_n < \frac{1}{c^{-n}}=c^n\xrightarrow[n\to\infty]{}0\]
$a_n>0$と合わせはさみうちの定理から$\displaystyle \lim_{n\to \infty}a_n=0$

$c=1$のとき$a_n=\frac{1}{2}\xrightarrow[n\to\infty]{}\frac{1}{2}$

$c>1$のとき$0<\frac{1}{c}<1$で$a_n=\frac{1}{(\frac{1}{c})^{-n}+(\frac{1}{c})^{n}}$なので$0<c<1$のときの結果より$\displaystyle \lim_{n\to \infty}a_n=0$

以上より$\displaystyle \lim_{n\to \infty}a_n=\begin{cases}
\frac{1}{2} & c=1\\
0 & c\neq 1\\
\end{cases}$

(vii)

$b_n=\frac{2\cdot4\cdot 6\cdots 2n}{3\cdot5\cdot 7\cdots (2n+1)}$とする.

$a_n>0,b_n>0$なので両方下に有界.

$\frac{2(n+1)-1}{2(n+1)}<1$より$a_{n+1}<a_n$

$\frac{2(n+1)}{2(n+1)+1}<1$より$b_{n+1}<b_n$

より$a_n,b_n$は両方単調減少で収束する.

それぞれ$a,b$に収束するとすると$a\geqq0,b\geqq 0$

$n\in\mathbb{N}-\{0\}$に対し$(2n)^2>(2n)^2-1\Leftrightarrow \frac{2n-1}{2n}<\frac{2n}{2n+1}$なので

$n=1,\cdots ,k$で掛けて$a_k<b_k$より$a\leqq b$

また$a_nb_n = \frac{1}{2n+1}\xrightarrow[n\to\infty]{}0$なので$ab=0$

$0\leqq a^2\leqq ab = 0$より$a=0$つまり$\displaystyle \lim_{n\to\infty}a_n=0$

2)

二項定理より
\[a_n=\sum_{k=0}^n \ _nC_k (\frac{1}{n})^k,a_{n+1}=\sum_{k=0}^{n+1} \ _{n+1}C_k (\frac{1}{n+1})^k\]
$l=0,\cdots ,k-1$に対し$n(n+1)-nl\geqq n(n+1)-(n+1)l\Leftrightarrow \frac{n+1-l}{n+1}\geqq \frac{n-l}{n}$

なので辺々掛けて$\frac{1}{k!}$で割り$ \ _nC_k (\frac{1}{n})^k \leqq \ _{n+1}C_k (\frac{1}{n+1})^k$
より
\[a_n=\sum_{k=0}^n \ _nC_k (\frac{1}{n})^k\leqq \sum_{k=0}^{n} \ _{n+1}C_k (\frac{1}{n+1})^k < \sum_{k=0}^{n+1} \ _{n+1}C_k (\frac{1}{n+1})^k=a_{n+1}\]
より$a_n$は単調増加.

また$ \ _nC_k\leqq \frac{n^k}{k!}$なので
\[a_n=\sum_{k=0}^n \ _nC_k (\frac{1}{n})^k\leqq \sum_{k=0}^{n} \frac{1}{k!}\]

$n\geqq3$で$\displaystyle \sum_{k=0}^n\frac{1}{k!} \leqq 2.9 - \frac{1}{n!}$を数学的帰納法で示す.

i) $n=3$のとき

$(左辺)=1+1+\frac{1}{2}+\frac{1}{6}<2.67 < 2.9-\frac{1}{6}=(右辺)$で成立.

ii) $n=l(\in \mathbb{N})$で成立すると仮定する.$(l\geqq 3)$

\[\sum_{k=0}^{l+1}\frac{1}{k!} \leqq 2.9-\frac{1}{l!}+\frac{1}{(l+1)!}=2.9-\frac{l}{(l+1)!}<2.9-\frac{1}{(l+1)!}\]

より$n=l+1$も成立.

i)ii)より示された.

より$a_n\leqq 2.9 -  \frac{1}{n!} < 2.9$で$a_n$は上に有界.より$a_n$は$e$に収束するとしてよく$e \leqq 2.9 < 3$

$n\geqq 2$で$a_n\geqq a_2 = \frac{9}{4}>2$
より$e>2$            

3)

$0 < a_n\leqq a_{n+1} \leqq b_{n+1} \leqq b_n$を数学的帰納法で示す.

i) $n=0$のとき

$a_0>0$だ.また$a<b$より$a<\frac{a+b}{2}<b$で$a_0 < a_1 < b_0$

より$\sqrt{a_1} < \sqrt{b_0}$で$a_1<\sqrt{a_1b_0}=b_1<b_0$

より成立.

ii) $n=k(\in\mathbb{N})$で成立すると仮定する.

$0<a_k\leqq a_{k+1}\leqq b_{k+1} \leqq b_k$

まず$a_{k+1}>0$.また$a_{k+1}\leqq b_{k+1}$より$a_{k+1}\leqq \frac{a_{k+1}+b_{k+1}}{2}=a_{k+2}\leqq b_{k+1}$

より$\sqrt{a_{k+2}} \leqq \sqrt{b_{k+1}}$で$a_{k+2}\leqq\sqrt{a_{k+2}b_{k+1}}=b_{k+2}\leqq b_{k+1}$

より$n=k+1$も成立.

i)ii)より示された.

より区間$[a_n,b_n]$は単調減少.また
\[b_{n+1}-a_{n+1}\leqq b_n-a_{n+1}=\frac{1}{2}(b_n-a_n)\]

これを繰り返し用い$b_n-a_n\leqq \frac{1}{2^n}(b_0-a_0)$

$\displaystyle \lim_{n\to\infty}\frac{1}{2^n}(b-a)=0,b_n-a_n\geqq 0$よりはさみうちの原理から$\displaystyle \lim_{n\to\infty}b_n-a_n=0$

以上より区間縮小法より$a_n,b_n$は収束し$\displaystyle \lim_{n\to \infty}a_n=\lim_{n\to\infty}b_n$で
この値を$l$とおける.

$0<a<b$より$0<x<\frac{\pi}{2}$としてよい.

$a_n = \frac{\sin x\cos \frac{x}{2^n}}{2^n\sin \frac{x}{2^n}}b,b_n= \frac{\sin x}{2^n\sin \frac{x}{2^n}}b$を数学的帰納法で示す.

i) $n=0$のとき

\[a_0=b\cos x = \frac{\sin x \cos \frac{x}{2^0}}{2^0\sin\frac{x}{2^0}}b,
b_0=b = \frac{\sin x}{2^0\sin\frac{x}{2^0}}b \]

より成立する.

ii) $n=k(\in \mathbb{N})$で成立すると仮定する.

\[a_{k+1}=\frac{a_k+b_k}{2}=\frac{\sin x}{2^k\sin \frac{x}{2^k}}b\cdot \frac{1+\cos \frac{x}{2^k}}{2}= \frac{\sin x\cos ^2 \frac{x}{2^{k+1}}}{2^k\sin \frac{x}{2^k}}b= \frac{\sin x\cos ^2 \frac{x}{2^{k+1}}}{2^{k+1}\sin \frac{x}{2^{k+1}}\cos \frac{x}{2^{k+1}}}b=\frac{\sin x\cos \frac{x}{2^{k+1}}}{2^{k+1}\sin \frac{x}{2^{k+1}}}b\]
\[b_{k+1}=\sqrt{a_{k+1}b_k}=\sqrt{\frac{\sin x\cos \frac{x}{2^{k+1}}}{2^{k+1}\sin \frac{x}{2^{k+1}}}b\cdot \frac{\sin x}{2^k\sin \frac{x}{2^k}}b}=\sqrt{\frac{\sin x\cos \frac{x}{2^{k+1}}}{2^{k+1}\sin \frac{x}{2^{k+1}}}b\cdot \frac{\sin x}{2^{k+1}\sin \frac{x}{2^{k+1}}\cos \frac{x}{2^{k+1}}}b}= \frac{\sin x}{2^{k+1}\sin \frac{x}{2^{k+1}}}b\]
より$n=k+1$も成立.

i)ii)より示された.

$\displaystyle \lim_{n\to\infty}\frac{x}{2^n}=0$より$\displaystyle \lim_{n\to\infty}\frac{\frac{x}{2^n}}{\sin \frac{x}{2^n}}=1$で
\[b_n= \frac{\sin x}{x}\cdot\frac{\frac{x}{2^n}}{\sin \frac{x}{2^n}}b\xrightarrow[n\to\infty]{} \frac{\sin x}{x}b\]
より$l=\frac{\sin x}{x}b$

(おまけ)

$a=\frac{1}{4},b=\frac{1}{2\sqrt{2}}$のとき$x=\frac{\pi}{4}$

直径$1$の円の中心を$O$,この円に外接,内接する正$2^{n+2}$角形の辺の1つをそれぞれ$AB,A'B'$とする.

また$AB,A'B'$の中点をそれぞれ$M,M'$とする.

\[\angle AOM = \angle A'OM' = \frac{2\pi}{2\cdot 2^{n+2}}=\frac{\pi}{2^{n+2}}\]
$MO,A'O$は円の半径で$\frac{1}{2}$
\[AM=MO\tan \frac{\pi}{2^{n+2}} = \frac{1}{2}\tan\frac{\pi}{2^{n+2}},A'M' =A'O\sin \frac{\pi}{2^{n+2}} = \frac{1}{2}\sin\frac{\pi}{2^{n+2}}\]
$AB=2AM,A'B'=2A'M'$で$2^{n+2}$個合わせてそれぞれ$2^{n+2}\tan\frac{\pi}{2^{n+2}},2^{n+2}\sin\frac{\pi}{2^{n+2}}$

逆数を取るとそれぞれ
\[\frac{1}{2^{n+2}\tan\frac{\pi}{ 2^{n+2}}}=\frac{\sin \frac{\pi}{4}\cos\frac{\pi}{2^{n+2}}}{2^n\sin\frac{\pi}{2^{n+2}}}\frac{1}{2\sqrt{2}}=\frac{\sin x\cos \frac{x}{2^n}}{2^n\sin\frac{x}{2^n}}b=a_n\]
\[\frac{1}{2^{n+2}\sin\frac{\pi}{ 2^{n+2}}}=\frac{\sin \frac{\pi}{4}}{2^n\sin\frac{\pi}{2^{n+2}}}\frac{1}{2\sqrt{2}}=\frac{\sin x}{2^n\sin\frac{x}{2^n}}b=b_n\]
また$l=\frac{\frac{1}{\sqrt{2}}}{\frac{\pi}{4}}\cdot \frac{1}{2\sqrt{2}}=\frac{1}{\pi}$

4)

$\mathbb{Q}$から順序体$K$への同型写像$g$を探す.

$x\in \mathbb{N}$に対し$g(x)=\begin{cases}
0_K & x=0\\
g(x-1)+1_K & x\neq 0\\ 
\end{cases}$とする.

$x,y\in\mathbb{N}$に対し$g(x+y)=g(x)+g(y)$を$y$に関する数学的帰納法で示す.

i) $y=0$のとき
\[g(x+0)=g(x)=g(x)+0_K=g(x)+g(0)\]
より成立.

ii) $y=k(\in\mathbb{N})$のとき成立すると仮定する.
\[g(x+k+1)=g(x+1)+g(k)=g(x)+1_K+g(k)=g(x)+g(k+1)\]
より$y=k+1$のときも成立.

i)ii)より$g(x+y)=g(x)+g(y)$

$x,y\in\mathbb{N}$に対し$g(xy)=g(x)g(y)$を$y$に関する数学的帰納法で示す.

i) $y=0$のとき
\[g(x0)=g(0)=0_K=g(x)0_K=g(x)g(0)\]
より成立.

ii) $y=k(\in\mathbb{N})$のとき成立すると仮定する.
\[g(x(k+1))=g(xk)+g(x)=g(x)g(k)+g(x)=g(x)(g(k)+1_K)=g(x)g(k+1)\]
より$y=k+1$のときも成立.

i)ii)より$g(xy)=g(x)g(y)$

$x\in\mathbb{N}-\{0\},y\in\mathbb{N}$に対し$g(y)< g(x+y)$を$x$に関する数学的帰納法で示す.

i) $x=1$のとき

$1_K > 0$より

$g(y)<1_K+g(y)=g(1)+g(y)=g(1+y)$より成立.

ii) $x=k(\in\mathbb{N}-\{0\})$のとき成立すると仮定する.

$1_K > 0$より
\[g(y) < g(k+y) < 1_K + g(k+y) = g(1) + g(k+y)  = g(k+1+y)\]
より$x=k+1$のときも成立.

i)ii)より$g(y) < g(x+y)$

$x,y\in\mathbb{N}$に対し$x < y$なら$y-x\in\mathbb{N}-\{0\}$で$g(x) < g(x+y-x) = g(y)$

$x=y$なら$g(x)=g(y)$

$x>y$なら$x<y$のときと同様に$g(x)>g(y)$

より$x\leqq y \Leftrightarrow g(x)\leqq g(y)$

以上より$g$は自然数に対して演算を保存.

$x\in \mathbb{Z}-\mathbb{N}$に対して$-x\in \mathbb{N}$で$g(x)=-g(-x)$と定義できる.

$x,y\in\mathbb{Z}$に対し$g(x+y)=g(x)+g(y)$を示す.

$x\geqq 0,y\geqq 0$は既に示した.

$x\geqq 0,y\leqq 0,x+y\geqq 0$のとき$g(x)=g(x+y)+g(-y)=g(x+y)-g(y)$より成立.

$x\geqq 0,y\leqq 0,x+y\leqq 0$のとき$g(y)=-g(-y)=-(g(-x-y)+g(x))=g(x+y)-g(x)$より成立.

$x\leqq 0,y\geqq 0$は$x\geqq 0,y\leqq 0$のときと同様.

$x\leqq 0,y\leqq 0$のとき$g(x)+g(y)=-(g(-x)+g(-y))=-g(-x-y)=g(x+y)$より成立.

より$g(x+y)=g(x)+g(y)$

次に$x,y\in\mathbb{Z}$に対し$g(xy)=g(x)g(y)$を示す.

$x\geqq 0,y\geqq 0$は既に示した.

$x\geqq 0,y\leqq 0$のとき$g(xy) = -g(x(-y))=-(g(x)g(-y))=g(x)g(y)$より成立.

$x\leqq 0,y\geqq 0$のとき$g(xy) = -g((-x)y)=-(g(-x)g(y))=g(x)g(y)$より成立.

$x\leqq 0,y\leqq 0$のとき$g(xy) = g((-x)(-y))=g(-x)g(-y)=g(x)g(y)$より成立.

より$g(xy)=g(x)g(y)$

$x\in \mathbb{Z},x \leqq 0$とすると$0 \leqq -x$より$g(0)\leqq g(-x)=-g(x)$で$g(x)\leqq g(0)=0_K$に注意する.

$x,y\in\mathbb{Z}$に対し$x\leqq y\Leftrightarrow g(x)\leqq g(y)$を示す.

$x\geqq 0,y\geqq 0$は既に示した.

$x\geqq 0,y\leqq 0$のとき
$y\leqq x,g(y)\leqq g(x)$が常に成り立ち成立.

$x\leqq 0,y\geqq 0$のとき$x\geqq 0,y\leqq 0$のときと同様に成立.

$x\leqq 0,y\leqq 0$のとき
\[g(x)\leqq g(y)\Leftrightarrow -g(x)\geqq -g(y) \Leftrightarrow g(-x)\geqq g(-y) \Leftrightarrow -x \geqq -y \Leftrightarrow x\leqq y\]
より成立.

より$x\leqq y\Leftrightarrow g(x)\leqq g(y)$

以上より$g$は整数に対して演算を保存.

$x\in \mathbb{Q}$とすると$x=\frac{p}{q}$とおける.\ $(q\in \mathbb{N}-\{0\},p\in\mathbb{Z})$

$q>0$なので$g(q)\neq 0_K$で$g(x) = \frac{g(p)}{g(q)}$と定義できる.

$q,s\in \mathbb{N}-\{0\},p,r\in\mathbb{Z}$に対し
\begin{align*}
&\frac{p}{q}=\frac{r}{s}\\
\Leftrightarrow & ps-qr=0\\
\Leftrightarrow & g(ps-qr)=0_K\\
\Leftrightarrow & \frac{g(ps-qr)}{g(qs)}=0_K \ (\because \frac{1_K}{g(qs)}\neq 0_K)\\
\Leftrightarrow & \frac{g(p)g(s)-g(q)g(r)}{g(q)g(s)}=0_K\\
\Leftrightarrow & \frac{g(p)}{g(q)}=\frac{g(r)}{g(s)}
\end{align*}
$\Leftarrow$から$g$がwell-definedなことが言え$\Rightarrow$から$g$が単射なことが言える.

また$x\in\mathbb{Z}$に対し$x=\frac{x}{1}$で$\frac{g(x)}{g(1)}=\frac{g(x)}{1_K}=g(x)$なので$g$の有理数での定義は整数での定義と矛盾しない.

$q,s\in \mathbb{N}-\{0\},p,r\in\mathbb{Z}$に対し
\begin{align*}
&g(\frac{p}{q}+\frac{r}{s})\\
= & g(\frac{ps+qr}{qs})\\
= & \frac{g(p)g(s)+g(q)g(r)}{g(q)g(s)}\\
= & \frac{g(p)}{g(q)}+\frac{g(r)}{g(s)}\\
= & g(\frac{p}{q})+g(\frac{r}{s})
\end{align*}

\begin{align*}
&g(\frac{p}{q}\cdot\frac{r}{s})\\
= & \frac{g(p)g(r)}{g(q)g(s)}\\
= & g(\frac{p}{q})g(\frac{r}{s})
\end{align*}

\begin{align*}
&\frac{p}{q}\leqq\frac{r}{s}\\
\Leftrightarrow & ps-qr\leqq0\\
\Leftrightarrow & g(ps-qr)\leqq 0_K\\
\Leftrightarrow & \frac{g(ps-qr)}{g(qs)}\leqq 0_K \ (\because \frac{1_K}{g(qs)}> 0_K)\\
\Leftrightarrow & \frac{g(p)g(s)-g(q)g(r)}{g(q)g(s)}\leqq 0_K\\
\Leftrightarrow & \frac{g(p)}{g(q)}\leqq\frac{g(r)}{g(s)}
\end{align*}
以上より$g$は有理数に対して演算を保存.

より$g$は$\mathbb{Q}$から$K$への準同型写像.

$g$を$\mathbb{Q}$から$g(\mathbb{Q})$への関数に制限すると$g$の単射性から同型写像.

つまり$g(\mathbb{Q})$は$\mathbb{Q}$と同型な順序体.

5)

$K,K'$は実数の性質がすべて成り立つので実数と同じように扱えることに注意する.

$K,K'$に対する自然数をそれぞれ$K_{\mathbb{N}},K'_{\mathbb{N}}$有理数をそれぞれ$K_{\mathbb{Q}},K'_{\mathbb{Q}}$とする.

この$K_{\mathbb{Q}},K'_{\mathbb{Q}}$がそれぞれ4)の$\mathbb{Q}$に同型な順序体であることは後で示しまずはこれを仮定して示す.

$K_{\mathbb{Q}} \cong \mathbb{Q} \cong K'_{\mathbb{Q}}$となり$K_{\mathbb{Q}}$から$K'_{\mathbb{Q}}$への同型写像$f$が存在.

$f$を$K$に拡張することを考える.

$x\in K$に対して定理3.9より$K_{\mathbb{Q}}$の数列で$x$に収束する$a_n$がある.

$\epsilon \in K'_{\mathbb{Q}}$が$\epsilon > 0_{K'}$のときアルキメデスの原理から$N'\in K'_{\mathbb{N}}$で$\epsilon > \frac{1_{K'}}{N'}$となるものが存在.

$f$は全単射なので$N\in K_\mathbb{Q}$で$f(N)=N'$となるものがただ1つ存在.$N' > 0_{K'}$より$N > 0_K$

$a_n$は収束列なのでコーシー列でもあり$M\in\mathbb{N}$が存在し$n,m\geqq M$で$|a_n-a_m|< \frac{1_K}{N}$

このとき$n,m\geqq M$で$|f(a_n)-f(a_m)|<\frac{1_{K'}}{f(N)}=\frac{1_{K'}}{N'}<\epsilon$

つまり$f(a_n)$はコーシー列で収束する.その収束先を$f(x)$とする.

この定義がwell-definedであることを言う.

$K_{\mathbb{Q}}$の数列$a_n,b_n$が共に$x$に収束するとして$f(a_n),f(b_n)$の収束先が同じであることを言えばいい.

$\epsilon ,N,N'$を上と同様に定義する.

$a_n-b_n$は$0_{K}$に収束するので$M\in\mathbb{N}$が存在し$n\geqq M$で$|a_n-b_n|<\frac{1_K}{N}$

$n\geqq M$で$|f(a_n)-f(b_n)|<\frac{1_{K'}}{N'}<\epsilon$

より$\displaystyle \lim_{n\to\infty}f(a_n)-f(b_n)=0_{K'}$で$\displaystyle \lim_{n\to\infty}f(a_n)=\displaystyle \lim_{n\to\infty}f(b_n)$

より$f$はwell-defined

$K_{\mathbb{Q}}$の数列$a_n,b_n$がそれぞれ$a,b(\in K)$に収束するとして$a<b$なら$f(a_n),f(b_n)$の収束先$f(a),f(b)$は$f(a)<f(b)$を満たすことを言う.

定理3.8を2度使い$a<x<y<b$となる$x,y\in K_{\mathbb{Q}}$が存在することが言える.

$n\geqq M$で$|a_n-a|<x-a,|b_n-b|<b-y$となる$M\in\mathbb{N}$が存在.

$n\geqq M$で$a_n<x<y<b_n$なので$f(a_n)<f(x)<f(y)<f(b_n)$

より$n\to\infty$として$f(a)\leqq f(x)<f(y) \leqq f(b)$で示せた.

特に$f$は単射.

$a<b$なら$f(a)<f(b)$で$a=b$なら$f(a)=f(b)$で$b<a$なら$f(b)<f(a)$より$a\leqq b\Leftrightarrow f(a)\leqq f(b)$

$K_{\mathbb{Q}}$の数列$a_n,b_n$がそれぞれ$a,b(\in K)$に収束するとすると$a_n+b_n$は$a+b$に$a_nb_n$は$ab$に収束する
\[f(a_n+b_n)=f(a_n)+f(b_n),f(a_nb_n)=f(a_n)f(b_n)\]
で$n\to\infty$とし
\[f(a+b)=f(a)+f(b),f(ab)=f(a)f(b)\]
以上より$f$は$K$から$K'$への同型写像.

最後に4)で存在を示した$K$の含む$\mathbb{Q}$と同型な体が$K_{\mathbb{Q}}$であることを言う.

以下$g$は4)で定義した$g$を表す.

$g(\mathbb{N}) = K_{\mathbb{N}}$を言えば$g$の整数,有理数への拡張の仕方から$g(\mathbb{Q}) = K_{\mathbb{Q}}$が言えるのでこれを言えばいい.

$g(\mathbb{N})$が帰納的集合であることを言う.

$g(0)=0_K$より$0_K\in g(\mathbb{N})$

$x\in g(\mathbb{N})$なら$g(y) = x$となる$y\in\mathbb{N}$が存在.$g(y+1)=x+1_K$で$x+1_K\in g(\mathbb{N})$

以上より$g(\mathbb{N})$が帰納的集合で$K_{\mathbb{N}}\subset g(\mathbb{N})$

$A=\{n\in \mathbb{N}|g(n)\in K_{\mathbb{N}}\}$としこれが帰納的集合であることを言う.

$g(0)=0_K\in K_{\mathbb{N}}$で$0\in A$

$x\in A$のとき$g(x)\in K_{\mathbb{N}}$で$g(x+1)=g(x)+1_K\in K_{\mathbb{N}}$より$x+1\in A$

より$A$は帰納的集合で$\mathbb{N}\subset A$

つまり$n\in\mathbb{N}$で$g(n)\in K_{\mathbb{N}}$

より$g(\mathbb{N}) \subset K_{\mathbb{N}}$

以上より$g(\mathbb{N}) = K_{\mathbb{N}}$

6)

(IV)と仮定する.$a\in A,b\in B$より$a<b$

$a < \frac{a+b}{2_K} < b$が言える.ただし$2_K=1_K+1_K$

$\frac{a+b}{2_K}\in K$なので$\frac{a+b}{2_K}\in A \cup B$

しかし$a$は$A$の最大元なので$\frac{a+b}{2_K}\notin A$

$b$は$B$の最小元なので$\frac{a+b}{2_K}\notin B$

より矛盾し(IV)となることはない.

7)

6)より連続の公理と順序体$K$の任意の切断$<A,B>$が(III)の形にならないことが同値であることを示せばいい.

$\Rightarrow$

$B\neq \emptyset$なので$b\in B$が存在.任意の$a\in A$に対し$a < b$なので$b$は$A$の上界.

より$A$は上に有界で$A\neq \emptyset$なので$A$の上限$s\in K$が存在.$K=A\cup B$なので$s\in A$か$s\in B$

$s \in A$と仮定する.$s$は$A$の上限なので特に上界で$a\in A$に対し$a\leqq s$より$s$は$A$の最大元.$A$に最大元が存在し(III)にはならない.

$s\in B$と仮定する.$b\in B$とすると任意の$a\in A$に対し$a< b$で$b$は$A$の上界.より$S$が$A$の上限なことより$s\leqq b$で$s$は$B$の最小元.$B$に最小元が存在し(III)にならない.

以上より示された.

$\Leftarrow$

 上に有界な$K$の部分集合$S(\neq\emptyset)$を考える.$S$は上に有界なので$U(\subset K)$を$S$の上界の集合とすると$U\neq \emptyset$
 
 $s\in S$が存在し$s-1_K\in K$で$s-1_K< s$より$s-1_K$は$S$の上界でない.より$D=\{x\in K|x\notin U\}$とすると$D\neq \emptyset$
 
 $x\in D$とすると$s\in S$が存在し$x < s$で$u\in U$に対し$s\leqq u$なので$x < u$
 
 $D\cup U = K,D\cap U=\emptyset$だ
 
 より仮定から(III)にならなく$D$に最大元が存在するか$U$に最小元が存在する.
 
 $D$に最大元が存在すると仮定する.最大元を$d$とすると$d\notin U$より$d < s$となる$s\in S$が存在.
 $\frac{s+d}{2_K}<s$より$\frac{s+d}{2_K}\in D$で$\frac{s+d}{2_K}>d$より$d$が最大元なことに矛盾.より背理法から$D$に最大元は存在しない.
 
 より$U$に最小元が存在する.つまり連続の公理が示された.

8)

問題文で定義された$R$を$R$教科書での定義の$R$を$\mathbb{R}$と表す.解析入門では$\mathbb{Q}$は$\mathbb{R}$から定義されているのでそれに従う.つまりこの問題では$\mathbb{R}$の性質を用いて$R$を得る.

$A$の要素は$\mathbb{Q}$のコーシー列だが$\mathbb{R}$の数列とみると収束列であることに注意する.

$0_A$を$0_A=(0)$,$-(a_n)=(-a_n)$とすると$A$が問題文で与えられた演算で可換環になることは有理数が体で特に可換環であることから示せる.

まず$R$の加算と乗算がwell-definedであることを言う.

$(a_n),(a'_n)$が同値で$(b_n),(b'_n)$も同値であるとする.$A$の要素はコーシー列なので
\[\lim_{n\to\infty}a_n=\lim_{n\to\infty}a'_n=a,\lim_{n\to\infty}b_n=\lim_{n\to\infty}b'_n=b \ (a,b\in\mathbb{R})\]
とできる.定理2.5より
\[\lim_{n\to\infty}((a_n+b_n)-(a'_n+b'_n))=(a+b)-(a+b)=0,\lim_{n\to\infty}(a_nb_n-a'_nb'_n)=ab-ab=0\]
より$(a_n+b_n)$と$(a'_n+b'_n),(a_nb_n)$と$(a'_nb'_n)$はそれぞれ同値でwell-defined

$A$が可換環なので$R$も可換環.

$(e_n)$を$e_n=1$となる数列とすると明らかに$(e_n)\in A$

$(a_n) \in A$が$[a_n]\neq [0]$としたとき$b_n=\begin{cases}
1 & a_n = 0\\
\frac{1}{a_n} & a_n\neq 0
\end{cases}$とする.

$\displaystyle \lim_{n\to\infty}a_n=a$とすると$[a_n]\neq [0]$より$a\neq 0$で$n\geqq M$で$|a_n-a|<|a|$となる$M\in\mathbb{N}$が存在.

$n\geqq M$で$a_n \neq 0$より$b_n=\frac{1}{a_n}\xrightarrow[n\to\infty]{} \frac{1}{a}$で$(b_n)$も$\mathbb{R}$のコーシー列.$b_n\in\mathbb{Q}$より$(b_n)\in A$

$a_nb_n=\begin{cases}
0 & a_n = 0\\
1 & a_n\neq 0
\end{cases}$で$n\geqq M$で$a_nb_n=1=e_n$

より$1_R=[e_n],[b_n]=[a_n]^{-1}$とすると$R$は(R-8),(R-9)を満たす.(R-10)は$\mathbb{Q}$の$1,0$が$1\neq 0$なことから$1_R\neq 0_R$で満たされる.

$R$の$[a_n]\leqq [b_n]$は問題文の$[a_n]<[b_n]$または$[a_n]=[b_n]$が成り立つという意味であることに注意する.

$[a_n]<[b_n]$なら$M(\in\mathbb{N}),\epsilon (>0)$が存在して$n\geqq M$で$a_n+\epsilon < b_n$より$n\to\infty$とし$\displaystyle \lim_{n\to\infty} a_n +\epsilon \leqq \lim_{n\to\infty}b_n$で$\displaystyle \lim_{n\to\infty} a_n < \lim_{n\to\infty}b_n$

また$\displaystyle \lim_{n\to\infty} a_n=a, \lim_{n\to\infty}b_n=b$としたとき$a<b$なら
$n\geqq M$で$|a_n-a|<\frac{b-a}{3},|b_n-b|<\frac{b-a}{3}$となる$M\in\mathbb{N}$が存在.

$n\geqq M$で$a_n<a+\frac{b-a}{3},b-\frac{b-a}{3}<b_n$で特に$a_n+\frac{b-a}{3}<b_n$

より$[a_n]<[b_n]$

以上より\[[a_n]<[b_n] \Leftrightarrow \lim_{n\to\infty}a_n<\lim_{n\to\infty}b_n\]

$[a_n]=[b_n]$なら$\displaystyle \lim_{n\to\infty}a_n = \lim_{n\to\infty}b_n$なので$R$の$<$はwell-defined

$a_n \in A$に対し$\displaystyle \lim_{n\to\infty}a_n \in \mathbb{R}$で$\mathbb{R}$は順序体の性質を満たすので$R$も順序体の性質を満たす.

$a,b\in R$を$a>0_R,b>0_R$とする.

$\frac{b}{a}\in R$でこれの代表元を$(q_n) (\in A)$とする.$n,m\geqq M$で$|q_n-q_m|<\frac{1}{2}$となる$M\in\mathbb{N}$が存在.

$n\geqq M$で$q_n < q_M+\frac{1}{2}$より$\displaystyle \lim_{n\to\infty} q_n \leqq q_M+\frac{1}{2} < q_M+1$

$q_M+1\in\mathbb{Q}$で$\frac{p}{q}$とおける.($q\in\mathbb{N}-\{0\},p\in\mathbb{Z}$)

$p\leqq 0$で$\displaystyle \lim_{n\to\infty} q_n < 0$で$\frac{b}{a} < 0_R\Leftrightarrow b < 0_Ra$

$p>0$で$\displaystyle \lim_{n\to\infty} q_n < p$で$\frac{b}{a} < (p)\Leftrightarrow b < (p)a$だ.ただし$(p)$はすべての自然数に対して$p$を返す数列.

$(p),0_R\in R_{\mathbb{N}}$よりアルキメデスの原理は成立.

$(r_n)$を$R$のコーシー数列とする.$(r_{in})$を$r_i$の代表元の有理数列とする.

$(r_{in})$の収束先を$d_i(\in\mathbb {R})$とする.

$n\geqq M_i$で$|r_{in} - d_i|<\frac{1}{i+1}$となる$M_i\in\mathbb{N}$が存在.有理数列$(a_i)$を$a_i=r_{iM_i}$と定義する.

$(r_n)$はコーシー列なので$\epsilon > 0$に対し$i,j\geqq M$で$|r_i-r_j| < \frac{\epsilon}{4_R}$となる$M\in\mathbb{N}$が存在.

このとき$|(r_{in})-(r_{jn})|<\frac{\epsilon}{4_R}\Leftrightarrow |d_i-d_j|<\frac{\epsilon}{4}$

$i,j\geqq max(M,\frac{4}{\epsilon})$で$|a_i-d_i|=|r_{iM_i}-d_i|<\frac{1}{i+1}<\frac{\epsilon}{4},|a_j-d_j|=|r_{jM_j}-d_j|<\frac{1}{j+1}<\frac{\epsilon}{4},|d_i-d_j|<\frac{\epsilon}{4}$

$|a_i-a_j|<\frac{3\epsilon}{4}<\epsilon$で$(a_n)$はコーシー列で$(a_n)\in A$

$i\geqq max(M,\frac{4}{\epsilon})$とし$|(r_{in})-(a_n)|$を考える.

$n\geqq max(M_i,i)$で$|a_n-d_n|=|r_{nM_n}-d_n|<\frac{1}{n+1}<\frac{\epsilon}{4},|d_i-d_n|<\frac{\epsilon}{4},|r_{in}-d_i|<\frac{1}{i+1}<\frac{\epsilon}{4}$

より$|a_n-r_{in}|<\frac{3\epsilon}{4}$で$n\to\infty$とし$\displaystyle |\lim_{n\to\infty}a_n - \lim_{n\to\infty} r_{in}| \leqq \frac{3\epsilon}{4}<\epsilon$つまり$|(a_n)-r_i|<\epsilon$

より$(r_n)$は$[a_n]$に収束.つまり$R$のコーシー列は収束する.より連続の公理も満たされる.

9)

$k_0+\frac{1}{k_1+\frac{1}{k_2+\cdots \frac{1}{k_n}}}$を$[k_0;k_1,k_2,\cdots,k_n]$と書く事にする.

$k_n>0 \ (n > 0)$を満たす整数列$(k_n)_{n\in\mathbb{N}}$に対し$(p'_n)_{n\in\mathbb{N}-\{0\}},(q'_n)_{n\in\mathbb{N}-\{0\}}$を以下のように定義する.
\[p'_n=\begin{cases}
k_0 & n=1 \\
k_0k_1+1 & n=2 \\
p'_{n-1}k_{n-1}+p'_{n-2} & n\geqq 3
\end{cases}, \ q'_n=\begin{cases}
1 & n=1 \\
k_1 & n=2 \\
q'_{n-1}k_{n-1}+q'_{n-2} & n\geqq 3
\end{cases}\]
$n\in \mathbb{N}-\{0\},t>1$に対し$[k_0;\cdots ,k_n,t]=\frac{p'_{n+1}t+p'_n}{q'_{n+1}t+q'_n}$を数学的帰納法で示す.

i) $n=1$のとき

\begin{align*}
[k_0;k_1,t]&=k_0+\frac{1}{k_1+\frac{1}{t}}\\
&=k_0+\frac{t}{tk_1+1}\\
&=\frac{(k_0k_1+1)t+k_0}{k_1t+1}\\
&=\frac{p'_2t+p'_1}{q'_2t+q'_1}
\end{align*}

ii) $n=m(\in \mathbb{N}-\{0\})$のときに成立すると仮定する.

\begin{align*}
[k_0;\cdots ,k_{m+1},t]&=[k_0;\cdots ,k_{m},k_{m+1}+\frac{1}{t}]\\
&=\frac{p'_{m+1}(k_{m+1}+\frac{1}{t})+p'_m}{q'_{m+1}(k_{m+1}+\frac{1}{t})+q'_m} \ (\because 帰納法の仮定)\\
&=\frac{(p'_{m+1}k_{m+1}+p'_m)t+p'_{m+1}}{(q'_{m+1}k_{m+1}+q'_m)t+q'_{m+1}}\\
&=\frac{p'_{m+2}t+p'_{m+1}}{q'_{m+2}t+q'_{m+1}}
\end{align*}

i)ii)より示された.

特に$t=k_{n+1}$として$[k_0;\cdots ,k_{n+1}]=\frac{p'_{n+1}k_{n+1}+p'_n}{q'_{n+1}k_{n+1}+q'_n}=\frac{p'_{n+2}}{q'_{n+2}}$

$[k_0;]=\frac{p'_1}{q'_1},[k_0;k_1]=\frac{p'_2}{q'_2}$と合わせて

$[k_0;\cdots ,k_n]=\frac{p'_{n+1}}{q'_{n+1}}$が任意の自然数で成り立つ.

次に$p'_{n+2}q'_{n+1}-p'_{n+1}q'_{n+2}=(-1)^n$を数学的帰納法で示す.

i) $n=0$のとき

$p'_2q'_1-p'_1q'_2=(k_0k_1+1)\cdot 1-k_0k_1=1=(-1)^0$で成立.

ii) $n=m(\in \mathbb{N})$のときに成立すると仮定する.
\begin{align*}
p'_{m+3}q'_{m+2}-p'_{m+2}q'_{m+3}&=(p'_{m+2}k_{m+2}+p'_{m+1})q'_{m+2}-p'_{m+2}(q'_{m+2}k_{m+2}+q'_{m+1})\\
&=-(p'_{m+2}q'_{m+1}-p'_{m+1}q'_{m+2})\\
&=(-1)^{m+1}
\end{align*}
で$n=m+1$も成立.

i)ii)より示された.

より$p'_n,q'_n$の公約数は1の約数で1.つまり$p'_n,q'_n$は互いに素.

$x\in \mathbb{R}-\mathbb{Q}$に対し問題文のように変数を設定すると$x$の連分数展開は$k_0=[x]$として$[k_0;\cdots,k_n,\cdots]$

上で示したことより$a_n=\frac{q'_n}{p'_n}$

$p'_n,q'_n$は互いに素なので$p_n=p'_n,q_n=q'_n$

より$x=\frac{p_nx_n+p_{n-1}}{q_nx_n+q_{n-1}}$

$q_n \geqq n-1$を数学的帰納法で示す.

i) $n=1,2,3$のとき

$q_1=1\geqq 0$

$k_1$は正整数なので$q_2=k_1\geqq 1$

$k_2$は正整数なので$q_3=q_2k_2+q_1\geqq 1\cdot 1 + 1 \geqq 2$

ii) $n=m,m+1,m+2 (m\in \mathbb{N}-\{0\})$のときに成立すると仮定する.

$k_{m+2}$は正整数なので$q_{m+3}=q_{m+2}k_{m+2}+q_{m+1} \geqq (m+1)\cdot 1 +m = 2m+1 \geqq m+2 $

で$n=m+3$も成立.

i)ii)より示された.

$n > 1$で
\begin{align*}
|x-a_n| &= |\frac{p_nx_n+p_{n-1}}{q_nx_n+q_{n-1}}-\frac{p_n}{q_n}|\\
&=|\frac{q_np_{n-1}-p_nq_{n-1}}{q_n(q_nx_n+q_{n-1})}|\\
&= |\frac{1}{q_n(q_nx_n+q_{n-1})}| \ (\because p_nq_{n-1}-p_{n-1}q_n=(-1)^{n-1})\\
&\leqq |\frac{1}{q_n^2}| \ (\because x_n >1 ,q_{n-1}\geqq n-2\geqq 0) 
\end{align*}
$n=0$も
\[|x-a_1|=x-[x]\leq 1=\frac{1}{q_1^2}\]
$\epsilon(>0)$に対し$N\in\mathbb{N}$が$N>\frac{1}{\sqrt{\epsilon}}+1$を満たすとする.

$n\geqq N$なら$q_n > \frac{1}{\sqrt{\epsilon}}$で$|x-a_n|\leqq \frac{1}{q_1^2}<\epsilon$

つまり$\displaystyle\lim_{n\to \infty}a_n=x$

(補足)

上の答案は無理数の連分数展開が無限に続くことを既知としている.その証明をここに書く.

$[k_0;k_1,k_2,\cdots,k_n]$が有理数なことを数学的帰納法で示す.ただし$k_i$は$i=0,\cdots, n$で整数で$k_i>0 \ (i>0)$

i) $n=0$のとき

$k_0\in\mathbb{Z}\subset \mathbb{Q}$
で成立.

ii) $n=m(\in\mathbb{N})$で成立すると仮定する.
\[[k_0;k_1,k_2,\cdots ,k_{m+1}]=k_0+\frac{1}{[k_1;k_2,\cdots ,k_{m+1}]}\]
帰納法の仮定より$[k_1;k_2,\cdots ,k_{m+1}]=\frac{q}{p} \ (p,q\in\mathbb{Z})$とおけ$[k_0;k_1,k_2,\cdots ,k_{m+1}]=\frac{k_0q+p}{q}$でこれは有理数.

より$n=m+1$のときも成立.

i)ii)より示された.

つまり$x$の連分数展開が有限なら$x$は有理数.

対偶を取り無理数の連分数展開は無限.

(おまけ)

有理数の連分数展開が有限の証明

$p\in\mathbb{Z},q\in \mathbb{N}-\{0\}$に対し$\frac{p}{q}$の連分数展開が有限であることを$q$に関する数学的帰納法で示す.

i) $q=1$のとき

$\frac{p}{q}=p$で連分数展開は$[p;]$より成立.

ii) $q=1,\cdots,m$で成立すると仮定する.

$p$が$m+1$の倍数のとき$\frac{p}{m+1}=[\frac{p}{m+1};]$

そうでないとき$p=(m+1)a+b$とおける. \ $(a\in\mathbb{Z} ,b=1,\cdots,m)$

\[\frac{p}{m+1}=a+\frac{1}{\frac{m+1}{b}}\]
帰納法の仮定から$\frac{m+1}{b}=[k_1;k_2,\cdots,k_n]$とおける.

$\frac{p}{m+1}=[a;k_1,k_2,\cdots,k_n]$で連分数展開は有限.

より$q=m+1$も成立.

i)ii)より有理数の連分数展開が有限

10)
\[\sqrt{2}-1 = \frac{1}{2+(\sqrt{2}-1)}\]
より$[a_0;a_1,\cdots ]$を$\sqrt{2}-1$の連分数展開とすると$a_0=0,a_1=2,a_n=a_{n-1}(n\geqq2)$

つまり$a_n=\begin{cases}
0 & n=0\\
2 & n\neq 0
\end{cases}$

より$\sqrt{2}$の連分数展開は$\begin{cases}
1 & n=0\\
2 & n\neq 0
\end{cases}$

\[\sqrt{3}-1=\frac{1}{1+\frac{\sqrt{3}-1}{2}},\frac{\sqrt{3}-1}{2}=\frac{1}{2+(\sqrt{3}-1)}\]
より$b_n,c_n$をそれぞれ$\sqrt{3}-1,\frac{\sqrt{3}-1}{2}$の連分数展開とすると
\[b_n=\begin{cases}
1 & n=1\\
c_{n-1} & n > 1
\end{cases},c_n = \begin{cases}
2 & n=1\\
b_{n-1} & n > 1
\end{cases}\]
より$b_1=1,b_2=2,b_n = b_{n-2} (n>2)$で$\sqrt{3}$の連分数展開は$\begin{cases}
1 & n=0\\
1 & nが正の奇数 \\
2 & nが正の偶数
\end{cases}$
\[\sqrt{5}-2=\frac{1}{4+(\sqrt{5}-2)}\]
より$d_n$を$\sqrt{5}-2$の連分数展開とすると$d_n=\begin{cases}
4 & n=1\\
d_{n-1} & n>1
\end{cases}$

つまり$d_n=4(n > 0)$で$\sqrt{5}$の連分数展開は$\begin{cases}
2 & n=0\\
4 & n\geq 0
\end{cases}$
\[\sqrt{6}-2=\frac{1}{2+\frac{\sqrt{6}-2}{2}},\frac{\sqrt{6}-2}{2}=\frac{1}{4+(\sqrt{6}-2)}\]
より$e_n,f_n$を$\sqrt{6}-2,\frac{\sqrt{6}-2}{2}$の連分数展開とすると\[e_n=\begin{cases}
2 & n=1\\
f_{n-1} & n>1
\end{cases},f_n=\begin{cases}
4 & n=1\\
e_{n-1} & n>1
\end{cases}\]
より$e_1=2,e_2=4,e_n = e_{n-2} (n>2)$で$\sqrt{6}$の連分数展開は$\begin{cases}
2 & n=0\\
2 & nが正の奇数 \\
4 & nが正の偶数
\end{cases}$
\[\sqrt{7}-2=\frac{1}{1+\frac{\sqrt{7}-1}{3}},\frac{\sqrt{7}-1}{3}=\frac{1}{1+\frac{\sqrt{7}-1}{2}},\frac{\sqrt{7}-1}{2}=\frac{1}{1+\frac{\sqrt{7}-2}{3}},\frac{\sqrt{7}-2}{3}=\frac{1}{4+(\sqrt{7}-2)}\]
より$g_n,h_n,k_n,l_n$を$\sqrt{7}-2,\frac{\sqrt{7}-1}{3},\frac{\sqrt{7}-1}{2},\frac{\sqrt{7}-2}{3}$の連分数展開とすると\[g_n=\begin{cases}
1 & n=1\\
h_{n-1} & n>1
\end{cases},h_n=\begin{cases}
1 & n=1\\
k_{n-1} & n>1
\end{cases},k_n=\begin{cases}
1 & n=1\\
l_{n-1} & n>1
\end{cases},l_n=\begin{cases}
4 & n=1\\
g_{n-1} & n>1
\end{cases}\]
より$g_1=1,g_2=1,g_3 =1, g_4=4,g_{n}=g_{n-4} (n>4)$で$\sqrt{7}$の連分数展開は$\begin{cases}
2 & n=0\\
1 & nが正で4の倍数でない \\
4 & nが正で4の倍数
\end{cases}$

11)

成り立たない有理数が存在するとしそれを$a$とする.$\epsilon (>0)$が存在し$x\in\mathbb{R}$に対し$|x-a|<\epsilon$なら$x\in\mathbb{Q}$

例えば$\sqrt{2}$は無理数なので無理数は存在し$p(\in \mathbb{R})$を無理数としてもいい.

$\mathbb{Q}$は$\mathbb{R}$で稠密なので$|q-p|<\epsilon$を満たす$q\in\mathbb{Q}$が存在.

$|(a+q-p)-a|=|q-p|<\epsilon$なので$a+q-p$は有理数.$a,q$は有理数で有理数は加算について閉じているので$p$も有理数となり矛盾.より背理法から示された.
\subsection*{$\S$4 $\mathbb{R}^n$と$\mathbb{C}$}
1)

i)を満たすことを言う.

$A,B\in\mathbb{R}^n$に対し$B-A\in \mathbb{R}^n$が唯一定まることを言えばいいが$\mathbb{R}^n$が加群であることから明らか.

ii)を満たすことを言う.

$a,A\in\mathbb{R}^n$に対し$B-A=a$を満たす$B\in\mathbb{R}^n$が唯一存在することを言えばいい.$B=A+a$のときのみ成立するのでii)も満たす.

iii)を満たすことを言う.

$A,B,C\in\mathbb{R}^n$に対し$(B-A)+(C-B)=C-A$なので成立.

以上より示された.

2)

$x,y\in\mathbb{R}^n$に対し
\[|g(x)-g(y)|=|(f(x)-f(0))-(f(y)-f(0))|=|f(x)-f(y)|=|x-y|\]
\[|g(x)|=|f(x)-f(0)|=|x-0|=|x|\]
\[|g(y)|=|f(y)-f(0)|=|y-0|=|y|\]
(4.15)より
\[|g(x)-g(y)|^2=|g(x)|^2-2(g(x)|g(y))+|g(y)|^2\]
\[|x-y|^2=|x|^2-2(x|y)+|y|^2\]
以上より
\[-2(g(x)|g(y))=-2(x|y)\Leftrightarrow (g(x)|g(y))=(x|y)\]
$(g(x)|g(y))=(x|y)$と命題4.2を用いると$x,y\in\mathbb{R}^n$に対し
\begin{align*}
& (g(x+y)-g(x)-g(y)|g(x+y)-g(x)-g(y))\\
=& (g(x+y)|g(x+y))+(g(x)|g(x))+(g(y)|g(y))-2(g(x)|g(x+y))-2(g(y)|g(x+y))+2(g(x)|g(y))\\
=& (x+y|x+y)+(x|x)+(y|y)-2(x|x+y)-2(y|x+y)+2(x|y)\\
=& ((x+y)-x-y|(x+y)-x-y)=(0|0)=0\\
\end{align*}
より命題4.2 4)より$g(x+y)-g(x)-g(y)=0$つまり$g(x+y)=g(x)+g(y)$

$(g(x)|g(y))=(x|y)$と命題4.2を用いると$x\in\mathbb{R}^n,a\in\mathbb{R}$に対し
\begin{align*}
& (g(ax)-ag(x)|g(ax)-ag(x))\\
=& (g(ax)|g(ax))+a^2(g(x)|g(x))-2a(g(x)|g(ax))\\
=& (ax|ax)+a^2(x|x)-2a(x|ax) = (a^2+a^2-2a^2)(x|x)=0(x|x) =0\\
\end{align*}
より命題4.2 4)より$g(ax)-ag(x)=0$つまり$g(ax)=ag(x)$

$e_i(\in \mathbb{R}^n)$を$i=1,\cdots,n$で以下のように定義する.
\[e_{ij} = \begin{cases}
1 & i=j\\
0 & i\neq j
\end{cases}\]
$x(\in\mathbb{R}^n)$は
\[x=\sum_{i=1}^nx_ie_i\]
なので$g(x+y)=g(x)+g(y),g(ax)=ag(x)$を用いて
\[g(x)=\sum_{i=1}^nx_ig(e_i)\]
$A\in\mathbb{R}^{n\times n}$を以下のように定義する.
\[A_{ij}=g(e_j)_i \ (i=1,\cdots,n,j=1,\cdots,n)\]
以下が$j=1,\cdots ,n$で成り立つ.
\[(Ax)_j = \sum_{i=1}^n A_{ji}x_i=\sum_{i=1}^nx_ig(e_i)_j=g(x)_j\]
より$Ax=g(x)$また
\[(A^tA)_{ij}=\sum_{k=1}^nA_{ki}A_{kj}=\sum_{k=1}^ng(e_i)_kg(e_j)_k=(g(e_i)|g(e_j))=(e_i|e_j)\]
ここで$(e_i|e_j)=\begin{cases}
1 & i = j\\
0 & i\neq j
\end{cases}$
なので$(A^tA)_{ij}=\begin{cases}
1 & i=j\\
0 & i\neq j
\end{cases}$

つまり$A^tA=I_n$で$A$は直交行列.

さらに$f(x)=g(x)+f(0)=Ax+f(0)$で$b=f(0)$とすると$f(x)=Ax+b$で示された.

逆に$A(\in\mathbb{R}^{n\times n})$が直交行列で$b\in\mathbb{R}^n$のとき$f(x)=Ax+b$とすると$x,y\in\mathbb{R}^n$に対し
\[|f(x)-f(y)|^2=|(Ax+b)-(Ay+b)|^2=|A(x-y)|^2=(x-y)^tA^tA(x-y)=(x-y)^tI_n(x-y)=(x-y)^t(x-y)=|x-y|^2\]
つまり$|f(x)-f(y)|=|x-y|$で$f$は合同変換.

3)

$x_1,\cdots,x_n$が1次独立なので$|x_1|\neq 0, |y_i|\neq 0 \ (i=2,\cdots,n)$に注意する.

$1\leqq i\leqq k,1\leqq j\leqq k$で$(u_i|u_j)=\delta_{i,j}$が$k=1,\cdots ,n$で成り立つことを$k$に関する数学的帰納法で示す.

i) $k=1$
\[(u_1|u_1)=\frac{(x_1|x_1)}{|x_1|^2}=1\]
より成立.

ii) $k=l$での成立を仮定する. $(1\leqq l \leqq n-1)$

$i=1,\cdots,l$で
\[(y_{l+1}|u_{i})=(x_{l+1}|u_i)-\sum_{j=1}^l(x_{l+1}|u_j)(u_j|u_i)=(x_{l+1}|u_i)-\sum_{j=1}^l(x_{l+1}|u_j)\delta_{ji}=(x_{l+1}|u_i)-(x_{l+1}|u_i)=0\]
より$(u_{l+1}|u_i)=\frac{(y_{l+1}|u_i)}{|y_{l+1}|}=0=\delta_{l+1,i}$

また$(u_i|u_{l+1})=(u_{l+1}|u_i)=0=\delta_{i,l+1}$

また$(u_{l+1}|u_{l+1})=\frac{(y_{l+1},y_{l+1})}{|y_{l+1}|^2}=1=\delta_{l+1,l+1}$

帰納法の仮定と合わせて$1\leqq i\leqq l+1,1\leqq j\leqq l+1$で$(u_i|u_j)=\delta_{i,j}$

以上より成立.
$k=n$として示された.

4)

$x,y\in\mathbb{R}^n$に対し
\begin{align*}|S_H(x)-S_H(y)|^2 &=|x-y-2\frac{(x-y|a)}{(a|a)}a|^2\\
&=|x-y|^2-4\frac{(x-y|a)}{(a|a)}(x-y|a)+4\frac{(x-y|a)^2}{(a|a)^2}(a|a)\\&=
|x-y|^2-4\frac{(x-y|a)^2}{|a|^2}+4\frac{(x-y|a)^2}{|a|^2}=|x-y|^2
\end{align*}
より$S_H$は$\mathbb{R}^n$の合同変換.

$x\in\mathbb{R}^n$に対し
\[(S_H(x)|a)=(x|a)-2((x|a)-c)=2c-(x|a)\]
よって
\[S_H^2(x)=S_H(x)-2((S_H(x)|a)-c)\frac{a}{(a|a)}=x-2((x|a)-c)\frac{a}{(a|a)}-2(c-(x|a))\frac{a}{(a|a)}=x\]
つまり$S_H^2=1$

5)
$x\in\mathbb{R}^n$に対して
\begin{align*}
d(x,a)=d(x,b)&\Leftrightarrow |x-a|^2=|x-b|^2\\
&\Leftrightarrow |x|^2-2(x|a)+|a|^2=|x|^2-2(x|b)+|b|^2\\
&\Leftrightarrow (x|a)-(x|b)-\frac{|a|^2-|b|^2}{2}=0\\
&\Leftrightarrow (a-b|x-\frac{a+b}{2})=0
\end{align*}
4)で$a\rightarrow a-b,c\rightarrow \frac{|a|^2-|b|^2}{2}$として
\[S_H(a)=a-2\left ((a|a-b)-\frac{|a|^2-|b|^2}{2}\right )\frac{a-b}{(a-b|a-b)}=a-2\left (a-b|\frac{a-b}{2}\right )\frac{a-b}{(a-b|a-b)}=a-(a-b)=b\]

6)

$x,y\in\mathbb{R}^n$に対し$z\in\mathbb{R}^n$を変数と見て
$d(z,x)=d(z,y)$は$xy$の垂直二等分超平面でこれを$H$とすると$|x|=|y|$より
\[S_H(0)=0-2\left ((0|x-y)-\frac{|x|^2-|y|^2}{2}\right )\frac{x-y}{(x-y|x-y)}=0-0=0\]
である.4)より$S_H$は合同変換なので2)より直交行列$A\in\mathbb{R}^{n\times n}$が存在し$Az+S_H(0)=S_H(z)$つまり$Az=S_H(z)$.また5)より$Ax=S_H(x)=y$

$|b-a|=|b'-a'|$なので$A(b-a)=b'-a'$を満たす直交行列$A$が存在する.

$g(x)=Ax-Aa+a'$とすると$g$は合同変換で$g(a)=a',g(b)=b'$

$g(c)=c'$なら$g$が条件を満たす.$g(c)\neq c$のときを考える.

$H'$を$c'g(c)$の垂直二等分超平面とする
$g$は合同変換なので
\[|g(c)-a'|=|c-a|=|c'-a'|\]
\[|g(c)-b'|=|c-b|=|c'-b'|\]
\[|g(c)|^2-2(g(c)|a')+|a'|^2=|c'|^2-2(c'|a')+|a'|^2\Leftrightarrow (c'-g(c)|a'-\frac{c'+g(c)}{2})=0\]
$S_H'(a')=a'-2(c'-g(c)|a'-\frac{c'+g(c)}{2})\frac{a}{(a|a)}=a'$で同様に$S_H'(b')=b'$

また5)より$S_H'(g(c))=c'$

以上より$S_H'(g(a))=a',S_H'(g(b))=b',S_H'(g(c))=c'$

$S_H',g$は合同変換なので$x,y\in\mathbb{R}^n$で
\[|S_H'(g(x))-S_H'(g(y))=|g(x)-g(y)|=|x-y|\]
つまり$S_H'(g(x))$は合同変換で示された.

7)

$\mathbb{R}^n$の独立な$n+1$個の点から等距離にある点が存在することを示す.

$n+1$個の点を$a_1,\cdots,a_{n+1}$とする.
$x\in\mathbb{R}^n$がこれらから等距離にある条件は
\begin{align*}
|x-a_i|=|x-a_{i+1}| &\Leftrightarrow |x-a_i|^2=|x-a_{i+1}|^2\\
&\Leftrightarrow |x|^2 -2(x|a_i) + |a_i|^2 = |x|^2 -2(x|a_{i+1}) + |a_{i+1}|^2\\
&\Leftrightarrow (x|a_i-a_{i+1}) = \frac{|a_i|^2-|a_{i+1}|^2}{2}
\end{align*}
が$i=1,\cdots,n$で成り立てばいい.

つまり$A\in\mathbb{R}^{n\times n},c\in \mathbb{R}^n$を以下のように定義して
\[A_{ij}=(a_i-a_{i+1})_j,c_i=\frac{|a_i|^2-|a_{i+1}|^2}{2}\]
$Ax=c$であればいい.$\det A \neq 0$なら$A^{-1}$が存在し$x = A^{-1}c$に決まる.

$\det A = 0$なら$Ay=0$となる$y\in\mathbb{R}^n$が存在するがこれは
\[(a_1|y)=\cdots=(a_{n+1}|y)\]
を意味する.つまり$a_1,\cdots,a_{n+1}$は以下の超平面上の点.
\[\{z\in\mathbb{R}^n|(z|y)=(a_1|y)\}\]
よって仮定に矛盾する.以上より示された.

8)

$0$と$b,c$は等距離にあるので
\[(c-b|h-a)=(c-b|c+b)=|c|^2-|b|^2=0\]
つまり$bc$は$ah$と直交する.

同様に$ab$は$ch$と$ca$は$bh$と直交する.

9)

8)と同様に$\triangle{abc}$を平行移動して$o=0$とする.$r$を正の実数として$|a|=|b|=|c|=r$とおける.

8)より$h=a+b+c$

$d=\frac{o+h}{2}=\frac{a+b+c}{2}$とする.

\[|d-\frac{b+c}{2}|=|\frac{a}{2}|=\frac{1}{2}|a|=\frac{r}{2}\]
同様にして$|d-\frac{c+a}{2}|=|d-\frac{a+b}{2}|=\frac{r}{2}$

$a$から$bc$に$b$から$ca$に$c$から$ab$に下ろした垂線の足をそれぞれ$a',b',c'$とする.

$a'$は直線$bc$上の点なので$t\in\mathbb{R}$として$a'=(1-t)b+tc$とおける.$aa'$と$bc$は直交するので
\[(c-b|(1-t)b+tc-a)=0\Leftrightarrow t=\frac{(c-b|a-b)}{|c-b|^2}\]
より
\begin{align*}
|d-a'|^2 &= |\frac{a+b+c}{2}-(1-t)b-tc|^2\\
&= \frac{1}{4}|a+(2t-1)b+(1-2t)c|^2\\
&= \frac{1}{4}(|a|^2+(2t-1)^2|c-b|^2 - (4t-2)(a|c-b))\\
&= \frac{1}{4}(|a|^2+(2t-1)(|c-b|^2(2t-1) - 2(a|c-b)))\\
&= \frac{1}{4}(|a|^2+(2t-1)(2|c-b|^2t-(c-b|c-b+2a)))\\
&= \frac{1}{4}(|a|^2+(2t-1)(2|c-b|^2t-(c-b|2a-2b))) \ (\because |b|=|c| \Leftrightarrow (c-b|c+b)=0)\\
&=\frac{|a|^2}{4}=\frac{r^2}{4}
\end{align*}
より
$|d-a'|=\frac{r}{2}$で同様にして$|d-b'|=|d-c'|=\frac{r}{2}$

また
\[|d-\frac{h+a}{2}|=\frac{|a|}{2}=\frac{r}{2}\]
同様にして$|d-\frac{h+b}{2}|=|d-\frac{h+c}{2}|=\frac{r}{2}$

以上より$\frac{a+b}{2},\frac{b+c}{2},\frac{c+a}{2},a',b',c',\frac{h+a}{2},\frac{h+b}{2},\frac{h+c}{2}$は$d$を中心とする半径$\frac{r}{2}$の円周上.

10)

一般に$B(a,r)\subset B(a',r')$なら$r' \geqq r+|a-a'|$を示す.$a=a'$なら明らかなので$a\neq a'$とする.

$|a-(a+r\frac{a'-a}{|a'-a|})|=|r\frac{a'-a}{|a'-a|}|=r$なので$a+r\frac{a'-a}{|a'-a|}\in B(a,r)$

仮定より$a+r\frac{a'-a}{|a'-a|}\in B(a',r')$で
$|a'-(a+r\frac{a'-a}{|a'-a|})|\leqq r'$が成立.

$|a'-(a+r\frac{a'-a}{|a'-a|})|=|\frac{(a'-a)}{|a'-a|}(r+|a'-a|)|=r+|a'-a|$

より$r'\geqq r+|a'-a|$で示された.

$B_i=(a_i,r_i)$とおける.

$B_{i+1}\subset B_i$なので$r_i \geqq r_{i+1} + |a_{i+1}-a_i|\geqq r_{i+1}$で$r_i$は単調減少.また$r_i\geqq 0$より$r_i$は下に有界.より$r_i$は$r$に収束するとしていい.

$r_i$は収束列なのでコーシー列でもあり$\epsilon > 0$に対して$j,k\geqq M$で$|r_j-r_k|<\epsilon $となる$M\in\mathbb{N}$が存在.

$j\geqq k\geqq M$で$B_j\subset B_k$で
\[r_k \geqq r_j + |a_j-a_k|\]
より
\[|a_j-a_k|\leqq r_k-r_j < \epsilon\]
同様にして$k\geqq j\geqq M$で$|a_j-a_k| < \epsilon$

以上より$j,k\geqq M$で$|a_j-a_k| < \epsilon$

より$a_i$はコーシー列で収束列でもあり$a$に収束するとしていい.
\[\bigcap_{m\in\mathbb{N}}B_m=B(a,r)\]
を示す.

$|x-a|>r$とする.

$j\geqq M$で$|a_j-a|<\frac{|x-a|-r}{2}$

$j\geqq M'$で$|r_j-r|<\frac{|x-a|-r}{2}$となる$M,M'\in\mathbb{N}$が存在.

$j\geqq max(M,M')$で
\[|x-a_j|\geqq |x-a|-|a-a_j|>|x-a|-\frac{|x-a|-r}{2}
= r + \frac{|x-a|-r}{2} > r + |r_j-r| \geqq r_j
\]
より$x\notin B_j$で特に$\displaystyle x\notin \bigcap_{m\in\mathbb{N}}B_m$

つまり$\displaystyle \bigcap_{m\in\mathbb{N}}B_m\subset B(a,r)$

$|x-a|<r$とする.

$j\geqq M$で$|a_j-a|<r - |x-a|$となる$M\in\mathbb{N}$が存在する.

$j\geqq M$で
\[|x-a_j| \leqq |x-a|-|a-a_j|<r\leqq r_j\]
つまり
$x\in B_j$

$B_0\supset B_1 \supset \cdots \supset B_M $ なので$j\in\mathbb{N}$で常に$x\in B_j$

より$\displaystyle x\in \bigcap_{m\in\mathbb{N}}B_m$

$\displaystyle \{x\in\mathbb{R}^n| |x-a|<r\}\subset \bigcap_{m\in\mathbb{N}}B_m$

$B_i$は閉集合なので$\displaystyle \bigcap_{m\in\mathbb{N}}B_m$も閉集合.

より$\displaystyle \overline{\{x\in\mathbb{R}^n| |x-a|<r\}} = B(a,r) \subset \bigcap_{m\in\mathbb{N}}B_m$

より$\displaystyle B(a,r) = \bigcap_{m\in\mathbb{N}}B_m$

で$\displaystyle \bigcap_{m\in\mathbb{N}}B_m$は閉球.

11)

$\mathbb{R}^4$は加群なので$\mathbb{H}$も$(R1)-(R4)$を満たし$0_\mathbb{H}=(0,0,0,0)$.

$(a_1,a_2,a_3,a_4),(b_1,b_2,b_3,b_4),(c_1,_2,c_3,c_4)\in\mathbb{H}$とする.
\begin{align*}
&(a_1,a_2,a_3,a_4)(b_1,b_2,b_3,b_4) \\
&= (a_1b_1-a_2b_2-a_3b_3-a_4b_4,a_1b_2+a_2b_1+a_3b_4-a_4b_3,a_1b_3+a_3b_1+a_4b_2-a_2b_4,a_1b_4+a_4b_1+a_2b_3-a_3b_2)\end{align*}
\begin{align*}
&(b_1,b_2,b_3,b_4)(c_1,c_2,c_3,c_4) \\
&= (b_1c_1-b_2c_2-b_3c_3-b_4c_4,b_1c_2+b_2c_1+b_3c_4-b_4c_3,b_1c_3+b_3c_1+b_4c_2-b_2c_4,b_1c_4+b_4c_1+b_2c_3-b_3c_2)\end{align*}
より$((a_1,a_2,a_3,a_4)(b_1,b_2,b_3,b_4))(c_1,c_2,c_3,c_4)$の第1成分は

$
(a_1b_1-a_2b_2-a_3b_3-a_4b_4)c_1-(a_1b_2+a_2b_1+a_3b_4-a_4b_3)c_2-(a_1b_3+a_3b_1+a_4b_2-a_2b_4)c_3-(a_1b_4+a_4b_1+a_2b_3-a_3b_2)c_4
$

$=a_1(b_1c_1-b_2c_2-b_3c_3-b_4c_4)
-a_2(b_2c_1+b_1c_2-b_4c_3+b_3c_4)
-a_3(b_3c_1+b_4c_2+b_1c_3-b_2c_4)
-a_4(b_4c_1-b_3c_2+b_2c_3+b_1c_4)
$

$=a_1(b_1c_1-b_2c_2-b_3c_3-b_4c_4)-a_2(b_1c_2+b_2c_1+b_3c_4-b_4c_3)-a_3(b_1c_3+b_3c_1+b_4c_2-b_2c_4)-
a_4(b_1c_4+b_4c_1+b_2c_3-b_3c_2)$

で$(a_1,a_2,a_3,a_4)((b_1,b_2,b_3,b_4)(c_1,c_2,c_3,c_4))$の第1成分と等しい.

より$((a_1,a_2,a_3,a_4)(b_1,b_2,b_3,b_4))(c_1,c_2,c_3,c_4)$の第2成分は

$
(a_1b_1-a_2b_2-a_3b_3-a_4b_4)c_2+(a_1b_2+a_2b_1+a_3b_4-a_4b_3)c_1+(a_1b_3+a_3b_1+a_4b_2-a_2b_4)c_4-(a_1b_4+a_4b_1+a_2b_3-a_3b_2)c_3
$

$=a_1(b_1c_2+b_2c_1+b_3c_4-b_4c_3)
-a_2(b_2c_2-b_1c_1+b_4c_4+b_3c_3)
-a_3(b_3c_2-b_4c_1-b_1c_4-b_2c_3)
-a_4(b_4c_2+b_3c_1-b_2c_4+b_1c_3)
$

$=a_1(b_1c_2+b_2c_1+b_3c_4-b_4c_3)+
a_2(b_1c_1-b_2c_2-b_3c_3-b_4c_4)+
a_3(b_1c_4+b_4c_1+b_2c_3-b_3c_2)-
a_4(b_1c_3+b_3c_1+b_4c_2-b_2c_4)$

で$(a_1,a_2,a_3,a_4)((b_1,b_2,b_3,b_4)(c_1,c_2,c_3,c_4))$の第2成分と等しい.

より$((a_1,a_2,a_3,a_4)(b_1,b_2,b_3,b_4))(c_1,c_2,c_3,c_4)$の第3成分は

$
(a_1b_1-a_2b_2-a_3b_3-a_4b_4)c_3+(a_1b_3+a_3b_1+a_4b_2-a_2b_4)c_1+(a_1b_4+a_4b_1+a_2b_3-a_3b_2)c_2-(a_1b_2+a_2b_1+a_3b_4-a_4b_3)c_4
$

$=a_1(b_1c_3+b_3c_1+b_4c_2-b_2c_4)
-a_2(b_2c_3+b_4c_1-b_3c_2+b_1c_4)
-a_3(b_3c_3-b_1c_1+b_2c_2+b_4c_4)
-a_4(b_4c_3-b_2c_1-b_1c_2-b_3c_4)
$

$=a_1(b_1c_3+b_3c_1+b_4c_2-b_2c_4)+
a_3(b_1c_1-b_2c_2-b_3c_3-b_4c_4)+
a_4(b_1c_2+b_2c_1+b_3c_4-b_4c_3)-
a_2(b_1c_4+b_4c_1+b_2c_3-b_3c_2)$

$(a_1,a_2,a_3,a_4)((b_1,b_2,b_3,b_4)(c_1,c_2,c_3,c_4))$の第3成分と等しい.

$((a_1,a_2,a_3,a_4)(b_1,b_2,b_3,b_4))(c_1,c_2,c_3,c_4)$の第4成分は

$
(a_1b_1-a_2b_2-a_3b_3-a_4b_4)c_4+(a_1b_4+a_4b_1+a_2b_3-a_3b_2)c_1+(a_1b_2+a_2b_1+a_3b_4-a_4b_3)c_3-(a_1b_3+a_3b_1+a_4b_2-a_2b_4)c_2
$

$=a_1(b_1c_4+b_4c_1+b_2c_3-b_3c_2)
-a_2(b_2c_4-b_3c_1-b_1c_3-b_4c_2)
-a_3(b_3c_4+b_2c_1-b_4c_3+b_1c_2)
-a_4(b_4c_4-b_1c_1+b_3c_3+b_2c_2)
$

$=a_1(b_1c_4+b_4c_1+b_2c_3-b_3c_2)+
a_4(b_1c_1-b_2c_2-b_3c_3-b_4c_4)+
a_2(b_1c_3+b_3c_1+b_4c_2-b_2c_4)-a_3(b_1c_2+b_2c_1+b_3c_4-b_4c_3)$

で$(a_1,a_2,a_3,a_4)((b_1,b_2,b_3,b_4)(c_1,c_2,c_3,c_4))$の第4成分と等しい.

以上より$((a_1,a_2,a_3,a_4)(b_1,b_2,b_3,b_4))(c_1,c_2,c_3,c_4)=(a_1,a_2,a_3,a_4)((b_1,b_2,b_3,b_4)(c_1,c_2,c_3,c_4))$で(R6)が成立.

また

$(a_1,a_2,a_3,a_4)((b_1,b_2,b_3,b_4)+(c_1,c_2,c_3,c_4))
$

$=(a_1,a_2,a_3,a_4)(b_1+c_1,b_2+c_2,b_3+c_3,b_4+c_4)$

$=(a_1b_1+a_1c_1-a_2b_2-a_2c_2-a_3b_3-a_3c_3-a_4b_4-a_4c_4,
a_1b_2+a_1c_2+a_2b_1+a_2c_1+a_3b_4+a_3c_4-a_4b_3-a_4c_3,
a_1b_3+a_1c_3+a_3b_1+a_3c_1+a_4b_2+a_4c_2-a_2b_4-a_2c_4,
a_1b_4+a_1c_4+a_4b_1+a_4c_1+a_2b_3+a_2c_3-a_3b_2-a_3c_2)$

$=(a_1b_1-a_2b_2-a_3b_3-a_4b_4,
a_1b_2+a_2b_1+a_3b_4-a_4b_3,
a_1b_3+a_3b_1+a_4b_2-a_2b_4,
a_1b_4+a_4b_1+a_2b_3-a_3b_2)+
(a_1c_1-a_2c_2-a_3c_3-a_4c_4,
a_1c_2+a_2c_1+a_3c_4-a_4c_3,
a_1c_3+a_3c_1+a_4c_2-a_2c_4,
a_1c_4+a_4c_1+a_2c_3-a_3c_2)$

$=(a_1,a_2,a_3,a_4)(b_1,b_2,b_3,b_4)+(a_1,a_2,a_3,a_4)(c_1,c_2,c_3,c_4)$

また

$((a_1,a_2,a_3,a_4)+(b_1,b_2,b_3,b_4))(c_1,c_2,c_3,c_4)
$

$=(a_1+b_1,a_2+b_2,a_3+b_3,a_4+b_4)(c_1,c_2,c_3,c_4)$

$=(a_1c_1+b_1c_1-a_2c_2-b_2c_2-a_3c_3-b_3c_3-a_4c_4-b_4c_4,
a_1c_2+b_1c_2+a_2c_1+b_2c_1+a_3c_4+b_3c_4-a_4c_3-b_4c_3,
a_1c_3+b_1c_3+a_3c_1+b_3c_1+a_4c_2+b_4c_2-a_2c_4-b_2c_4,
a_1c_4+b_1c_4+a_4c_1+b_4c_1+a_2c_3+b_2c_3-a_3c_2-b_3c_2)$

$=(a_1c_1-a_2c_2-a_3c_3-a_4c_4,
a_1c_2+a_2c_1+a_3c_4-a_4c_3,
a_1c_3+a_3c_1+a_4c_2-a_2c_4,
a_1c_4+a_4c_1+a_2c_3-a_3c_2)+
(b_1c_1-b_2c_2-b_3c_3-b_4c_4,
b_1c_2+b_2c_1+b_3c_4-b_4c_3,
b_1c_3+b_3c_1+b_4c_2-b_2c_4,
b_1c_4+b_4c_1+b_2c_3-b_3c_2)$

$=(a_1,a_2,a_3,a_4)(c_1,c_2,c_3,c_4)+
(b_1,b_2,b_3,b_4)(c_1,c_2,c_3,c_4)$

以上より(R7)も成立.

$(a_1,a_2,a_3,a_4)(1,0,0,0)=(a_1,a_2,a_3,a_4)$

よって(R8)は成立し$1_\mathbb{H}=(1,0,0,0)$

$(a_1,a_2,a_3,a_4)(a_1,-a_2,-a_3,-a_4)$

$=(a_1a_1+a_2a_2+a_3a_3+a_4a_4,
-a_1a_2+a_2a_1-a_3a_4+a_4a_3,
-a_1a_3+a_3a_1-a_4a_2+a_2a_4,
-a_1a_4+a_4a_1-a_2a_3+a_3a_2)$

$=(a_1^2+a_2^2+a_3^2+a_4^2,0,0,0)$

より$(a_1,a_2,a_3,a_4)\neq (0,0,0,0)$で

\[(a_1,a_2,a_3,a_4)(\frac{a_1}{a_1^2+a_2^2+a_3^2+a_4^2},-\frac{a_2}{a_1^2+a_2^2+a_3^2+a_4^2},-\frac{a_3}{a_1^2+a_2^2+a_3^2+a_4^2},-\frac{a_4}{a_1^2+a_2^2+a_3^2+a_4^2})=(1,0,0,0)=1_\mathbb{H}\]

つまり(R9)は成立.

$(1,0,0,0)\neq (0,0,0,0)$で(R10)は成立.

以上より示された.

12)

代数学の基本定理より$x\in\mathbb{C},a_0,\cdots,a_n\in\mathbb{R}$で$a_n\neq 0$として
\[\sum_{k=0}^na_kx^k= a_n(x-\alpha_1)\cdots (x-\alpha_n)\]
となる$\alpha_1,\cdots ,\alpha_n\in\mathbb{C}$が存在する.

\[f(x)=\sum_{k=0}^na_kx^k\]

とすると$k\in\mathbb{N},p,q\in\mathbb{R}$で$f^{(k)}(p+qi)=\overline{f^{(k)}(p-qi)}$なので$f^{(k)}(p+qi)=0\Leftrightarrow f^{(k)}(p-qi)=0$

よって$p+qi$と$p-qi$で$\alpha_j$に含まれる数は等しい.

より$\beta_j,\gamma_j\in\mathbb{R},\gamma_j >0$として
$\alpha_1=\beta_1+\gamma_1i,\alpha_2=\beta_1-\gamma_1i,\cdots ,\alpha_{2k-1}=\beta_k+\gamma_ki,\alpha_{2k}=\beta_k-\gamma_ki$で$\alpha_{2k+1},\cdots,\alpha_{n} \in\mathbb{R}$としていい.このとき
\[f(x)=a_n(x^2-2\beta_1+\beta_1^2+\gamma_1^2)\cdots (x^2-2\beta_k+\beta_k^2+\gamma_k^2)(x-\alpha_{2k+1})\cdots (x-\alpha_n)\]

ところで$x,a_j,b_j,c_j\in\mathbb{R}$で
\[(\sum_{k=0}^n a_kx^k)(\sum_{k=0}^m b_kx^k)=\sum_{k=0}^{n+m} c_kx^k\]
なら
\[c_k = \sum_{i+j=k}a_ib_j\]である.
$x\in K$でも$x^i\in K,b_j\in\mathbb{R}$より$x^ib_j=b_jx^i$なので
\[(\sum_{k=0}^n a_kx^k)(\sum_{k=0}^m b_kx^k)=\sum_{k=0}^{n+m}\sum_{i+j=k}a_ix^ib_jx^j=\sum_{k=0}^{n+m}\sum_{i+j=k}a_ib_jx^{i+j}=\sum_{k=0}^{n+m}c_kx^{k}\]
これを繰り返すと
\[f(x)=a_n(x^2-2\beta_1+\beta_1^2+\gamma_1^2)\cdots (x^2-2\beta_k+\beta_k^2+\gamma_k^2)(x-\alpha_{2k+1})\cdots (x-\alpha_n)\]
は$x\in K$でも成り立つ.

ここで$x\in K$とすると$K$は有限次元なので$\{1,x,\cdots,x^n\}$が線形独立でなくなる$n$が存在しそのうち最小のものを取ると
\[\sum_{k=0}^na_kx^k = 0_K\]
となる$a_k\in\mathbb{R}$が存在し$a_n\neq 0$となる.よって
\[a_n(x^2-2\beta_1+\beta_1^2+\gamma_1^2)\cdots (x^2-2\beta_k+\beta_k^2+\gamma_k^2)(x-\alpha_{2k+1})\cdots (x-\alpha_n)=0_K\]
としていい($\alpha_j,\beta_j,\gamma_j\in\mathbb{R}$).

$(x^2-2\beta_1+\beta_1^2+\gamma_1^2),\cdots ,(x^2-2\beta_k+\beta_k^2+\gamma_k^2),(x-\alpha_{2k+1}),\cdots ,(x-\alpha_n)$が全て$0_K$でないとすると(R9)を使いそれぞれの項の逆元を逆順に右から掛けると$a_n=0_K$となり矛盾する.

よりいずれかは$0_K$になる.

$x-\alpha=0_K$または$x^2-2\beta+\beta^2+\gamma^2 = 0_K$としていい($\alpha ,\beta, \gamma\in\mathbb{R},\gamma >0$).

ところで
\[I=\{x\in K|x^2\in\mathbb{R},x^2 < 0\}\]
とする.$x-\alpha=0_K$なら$x\in\mathbb{R}$で$x^2-2\beta+\beta^2+\gamma^2 = 0_K$なら$(x-\beta)^2=-\gamma^2 < 0$で
$x\notin \mathbb{R}$なら
$x=y+z,y\in\mathbb{R},z\in I$とおける.

$I$の要素$x$で$x^2=-1$となるものが存在しないとき

$x^2-2\beta+\beta^2+\gamma^2 = 0_K$なら$(\frac{x-\beta}{\gamma})^2=-1$となり矛盾するので$x=\alpha\in\mathbb{R}$で$K=\mathbb{R}$

$I$の要素$x$で$x^2=-1$となるものが存在するときその1つを$i$とすると$(-i)(-i)=(-1)(-1)ii=-1$で$-i$も満たす.これ以外に存在しないとすると

$x^2-2\beta+\beta^2+\gamma^2 = 0_K$なら$\frac{x-\beta}{\gamma}=\pm i$で$x=\beta \pm \gamma i$

$x-\alpha=0_K$も合わせて$x=y+zi \ (y,z\in\mathbb{R})$と表せる.逆に$1,i\in K$で$K$はベクトル空間なので$y,z\in\mathbb{R}$で$y+zi$は$K$の要素.よりこのとき
$K$は$\mathbb{C}$と同型だ.

$j\in K$で$j^2=-1,j\neq i,j\neq -i$が存在するとする.

$a,b\in\mathbb{R}$として$a+bi=0_K$とする.

$a=-bi$で$a^2=(-bi)^2=-b^2$だ.$a^2+b^2=0$なので$a=b=0$

つまり$\{1,i\}$は1次独立.

$a,b,c\in\mathbb{R}$として$a+bi+cj=0_K$とする.

$a+bi=-cj$で$a^2-b^2+2abi=-c^2$

$\{1,i\}$は1次独立なので$a^2-b^2+c^2=0,2ab=0$

$b=0$なら$a^2+c^2=0$で$a=b=c=0$

$b\neq 0$なら$a=0$で$c=\pm b$

$a+bi+cj=b(i\pm j)$で$b\neq 0,i\neq j,i\neq -j$より$a+bi+cj\neq 0_K$で不適.

以上より$a=b=c=0$で$\{1,i,j\}$は1次独立.

$a,b,c,d\in\mathbb{R}$として$a+bi+cj+dij=0_K$とする.

$d=0$なら$\{1,i,j\}$は1次独立なので$a=b=c=d=0$

$d\neq 0$なら$a'=-\frac{a}{d},b'=-\frac{b}{d},c'=-\frac{c}{d}$として$ij=a'+b'i+c'j$
\[-j=iij=a'i-b'+c'ij
=a'i-b'+c'(a'+b'i+c'j)=-b'+a'c'+(a'+b'c')i+c'^2j
\]
$\{1,i,j\}$は1次独立なので$j$の係数を
比較して$c'^2+1=0$となり$c'=-\frac{c}{d}\in\mathbb{R}$に矛盾し不適.

以上より$a=b=c=d=0$で$\{1,i,j,ij\}$は1次独立.

次に$ij\in K$で$\{1,ij\}$は1次独立なので$ij\notin\mathbb{R}$で$ij=x+y \ (x\in\mathbb{R},y\in I)$とおける.

$y^2=e$とすると$e\in\mathbb{R}$で$e<0$

$(ij-x)^2=e$なので$x^2 -e - 2ijx+ijij=0_K$

両辺に$ji$を右から掛けて$(x^2-e)ji-2x+ij=0_K$

$x^2-e > 0$なので$ji=\frac{2x}{x^2-e}-\frac{1}{x^2-e}ij$

$X=\frac{2x}{x^2-e},Y=-\frac{1}{x^2-e}$として$ji=X+Yij$.
\[-j=jii=Xi+Yiji=Xi+Yi(X+Yij)=X(1+Y)i-Y^2j\]
なので$\{i,j\}$は1次独立だから
\[X(1+Y)=0,Y\pm 1\]

$Y=-1$または$X=0,Y=1$

$X=0,Y=1$なら$ij=ji$で
\[(i+j)(i-j)= i^2-j^2-ij+ji=-1-(-1)-(ij-ji)=0_K\]
$i\neq j,i\neq -j$なので矛盾.

より$Y=-1$で$ij+ji=X$

このとき$x^2-e=1$で$e<0$より$-1<x<1$

$X=\frac{2x}{x^2-e}=\frac{2x}{1}=2x$なので$-2<X<2$
\[i\frac{Xi+2j}{\sqrt{4-X^2}}+\frac{Xi+2j}{\sqrt{4-X^2}}i=\frac{-2X+2(ij+ji)}{\sqrt{4-X^2}}=0_K\]
$(\frac{Xi+2j}{\sqrt{4-X^2}})^2=-1$で
$i\neq \frac{Xi+2j}{\sqrt{4-X^2}},i\neq -\frac{Xi+2j}{\sqrt{4-X^2}}$なので$j\rightarrow \frac{Xi+2j}{\sqrt{4-X^2}}$としてもよくこのとき$ij+ji=0_K$

$ij=k$とすると$k^2=ijij=-iijj=-1$

また$ik=iij=-j,jk=jij=-jji=i,ki=iji=-iij=j,kj=ijj=-i$

$l^2=-1$とすると$a,b,c\in\mathbb{R}$として
$l=ai+bj+ck$と表せることを示す.表せないと仮定して矛盾を導く.

上と同様の議論で$l=\pm i$か$il+li=x \ x\in\mathbb{R}$

$l=ai+bj+ck$と表せないので$il+li=x$

同様にして$jl+lj=y,kl+lk=z \ y,z\in\mathbb{R}$

\[z-lk=kl=ijl=i(y-lj)=iy-ilj=iy-(x-li)j=iy-xj+lij=yi-xj+lk\]

より$k=ij$に注意し
\[l=(-yi+xj+z)(2k)^{-1}=\frac{1}{2}(-yi+xj+z)(-ij)=
-\frac{x}{2}i-\frac{y}{2}j-\frac{z}{2}k\]
となり$l=ai+bj+ck$と表せないことに矛盾.

背理法から$l=ai+bj+ck$とおける.

$x^2-2\beta x+\beta^2+\gamma^2=0_K$なら$(\frac{(x-\beta)}{\gamma})^2=-1$で$\frac{(x-\beta)}{\gamma}=ai+bj+ck\Leftrightarrow x = \beta+\gamma ai+\gamma bj+\gamma ck$

$x-\alpha=0_K$なら$x=\alpha$

いずれの場合も$x=a+bi+cj+dk \ (a,b,c,d\in\mathbb{R})$とおける.

逆に$1,i,j,ij\in K$で$K$はベクトル空間なので$a,b,c,d\in\mathbb{R}$で$a+bi+cj+dk$は$K$の要素.

$\{1,i,j,k\}$は1次独立だから
$a+bi+cj+dk$は$(a,b,c,d) \in\mathbb{H}$と1対1対応する.

$(a_1+a_2i+a_3j+a_4k)(b_1+b_2i+b_3j+b_4k)
=a_1b_1-a_2b_2-a_3b_3-a_4b_4+(a_1b_2+a_2b_1+a_3b_4-a_4b_3)i
(a_1b_3+a_3b_1+a_4b_2-a_2b_4)j
(a_1b_4+a_4b_1+a_2b_3-a_3b_2)k$

$(a_1+a_2i+a_3j+a_4k)+(b_1+b_2i+b_3j+b_4k)=a_1+b_1+(a_2+b_2)i+(a_3+b_3)j+(a_4+b_4)k$

なので$a+bi+cj+dk$から$(a,b,c,d)\in\mathbb{H}$への対応は同型写像.

以上より$K=\mathbb{R},\mathbb{C},\mathbb{H}$
のいずれかと同型である.

\subsection*{$\S$5 級数}
1)

$r < 1$なら$\frac{1-r}{2}>0$で$\displaystyle \lim_{n\to\infty} \sqrt[n]{a_n}=r$なので$n\geqq n_0$で$|\sqrt[n]{a_n}-r|<\frac{1-r}{2}$となる$n_0\in\mathbb{N}$が存在する.

$n\geqq n_0$で$\sqrt[n]{a_n} < r+ \frac{1-r}{2}=\frac{r+1}{2}$で$\frac{r+1}{2}<1$なので定理5.6 1)より$\sum a_n$は収束する.

$r > 1$なら$\frac{r-1}{2}>0$で同様にして$n\geqq n_1$で$|\sqrt[n]{a_n}-r|<\frac{r-1}{2}$となる$n_1\in\mathbb{N}$が存在する.

$n\geqq n_1$で$\sqrt[n]{a_n} > r- \frac{r-1}{2}=\frac{r+1}{2}$で$\frac{r+1}{2}>1$なので定理5.6 2)より$\sum a_n$は発散する.

2)

(i)
$\sum \frac{2n^2}{n^3+1},\sum \frac{1}{n}$は共に正項級数で$n\geqq 1$で$\frac{2n^2}{n^3+1} > \frac{2n^2}{n^3+n^3} = \frac{1}{n}$

例4より$\displaystyle \sum_{n=1}^\infty \frac{1}{n}$は発散するので定理5.5より$\sum \frac{2n^2}{n^3+1}$は発散する.

(ii)

$n>0$で
$\frac{\sqrt{n}}{1+n^2} < \frac{\sqrt{n}}{n^2} = \frac{1}{n^{\frac{3}{2}}}$

定理V 2.5より $\displaystyle\sum_{n=1}^{\infty} \frac{1}{n^{\frac{3}{2}}}$は収束する.

より定理5.5から$\sum \frac{\sqrt{n}}{1+n^2}$は収束する.

(iii)

$a > 1$のとき$x > 0$で$e^x > 1+x$なので$x\rightarrow \frac{\log a}{n}$として$a^{\frac{1}{n}}-1 > \frac{\log a}{n}$.

例4より$\displaystyle \sum_{n=1}^\infty \frac{1}{n}$は発散するので定理5.5より$\sum (a^{\frac{1}{n}}-1)$は発散する.

$a = 1$のとき$a^{\frac{1}{n}}-1=0$で$\sum (a^{\frac{1}{n}}-1)=0$で収束.

$a < 1$のとき$b=\frac{1}{a}$として$b^{\frac{1}{n}} > \frac{\log b}{n} + 1$なので$n\geqq 1$で
\[1-a^{\frac{1}{n}}  = 1-\frac{1}{b^{\frac{1}{n}}} > 1-\frac{1}{\frac{\log b}{n} + 1} =
\frac{\frac{\log b}{n}}{\frac{\log b}{n} + 1}
\geqq \frac{1}{n}\cdot\frac{\log b}{\log b+1}\]
上と同様にして$\sum (1- a^{\frac{1}{n}})$は発散する.つまり$\sum (a^{\frac{1}{n}}-1)$は発散する.

以上より$\begin{cases}
発散する & a\neq 1\\
収束する & a = 1
\end{cases}$

(iv)

$b_n=\frac{1}{n^2},c_n = \frac{(-1)^n}{n}$とすると例5,例7より$\sum b_n,\sum c_n$は収束する.それぞれ$b,c\in\mathbb{R}$に収束するとする.

$\epsilon > 0$に対して$M_1,M_2\in\mathbb{N}$が存在し
\[n\geqq M_1\Rightarrow |\sum_{k=1}^nb_k - b| < \frac{\epsilon}{2} , n\geqq M_2\Rightarrow |\sum_{k=1}^nc_k - c| < \frac{\epsilon}{2} \]
$n\geqq 2\max (M_1,M_2)+2$で$n$が偶数の時
\begin{align*}
|\sum_{k=2}^n a_k - (b+c)| &= |(\sum_{k=1}^{\frac{n}{2}} b_k - b)+(\sum_{k=1}^{\frac{n-2}{2}} c_k - c)|\\
&\leqq |\sum_{k=1}^{\frac{n}{2}} b_k - b|+|\sum_{k=1}^{\frac{n-2}{2}} c_k - c| < \frac{\epsilon}{2}+\frac{\epsilon}{2} = \epsilon \ (\because \frac{n-2}{2} \geqq M_1,M_2)
\end{align*}
$n\geqq 2\max (M_1,M_2)+1$で$n$が奇数の時
\begin{align*}
|\sum_{k=2}^n a_k - (b+c)| &= |(\sum_{k=1}^{\frac{n-1}{2}} b_k - b)+(\sum_{k=1}^{\frac{n-1}{2}} c_k - c)|\\
&\leqq |\sum_{k=1}^{\frac{n-1}{2}} b_k - b|+|\sum_{k=1}^{\frac{n-1}{2}} c_k - c| < \frac{\epsilon}{2}+\frac{\epsilon}{2} = \epsilon \ (\because \frac{n-1}{2} \geqq M_1,M_2)
\end{align*}
以上より$n\geqq 2\max (M_1,M_2)+2$で$\displaystyle |\sum_{k=2}^na_k - (b+c)| < \epsilon$

より$\sum a_n$は$b+c$に収束する.

(v)

$a_n=\frac{n}{2^n}$とすると$\sum a_n$は正項級数で
\[\frac{a_{n+1}}{a_n}=\frac{n+1}{2n}=\frac{1+\frac{1}{n}}{2}\xrightarrow[n\to\infty]{}\frac{1}{2}<1\]
よって定理5.7より$\sum a_n$は収束する.

(vi)

$a_n = \begin{cases}
\frac{1}{n!} & nが奇数 \\
-\frac{1}{n} & nが偶数 
\end{cases}$とする.

$b_n=\frac{1}{(2n-1)!}$とすると
\[\frac{b_{n+1}}{b_n}=\frac{1}{2n(2n+1)}\xrightarrow[n\to\infty]{}0<1\]
よって定理5.7より$\sum b_n$は収束する.収束値を$b$とする.

$c_n=\frac{1}{2n}$とすると例4より$\displaystyle \sum_{n=1}^\infty \frac{1}{n}$は発散するので$\sum c_n$も発散する.$c_n$は正項級数なので正の無限大に発散する.

$M > 0$に対して$N_1,N_2\in\mathbb{N}$が存在し
\[n\geqq N_1 \Rightarrow |\sum_{k=1}^nb_k-b| < \frac{M}{2},n\geqq N_2 \Rightarrow \sum_{k=1}^nc_k > \frac{3M}{2}+b\]
$n\geqq \max (2N_1-1,2N_2+1)$で$n$が奇数の時
\[\sum_{k=1}^n a_k = \sum_{k=1}^{\frac{n+1}{2}} b_k - \sum_{k=1}^{\frac{n-1}{2}}c_k
< (\frac{M}{2}+b) - (\frac{3M}{2}+b) < -M\]
$n\geqq \max (2N_1,2N_2)$で$n$が奇数の時
\[
\sum_{k=1}^n a_k = \sum_{k=1}^{\frac{n}{2}} b_k - \sum_{k=1}^{\frac{n}{2}}c_k\\
< (\frac{M}{2}+b) - (\frac{3M}{2}+b) < -M
\]
以上より$n\geqq \max (2N_1,2N_2+1)$で$\displaystyle\sum_{k=1}^n a_k < -M$
より$1-\frac{1}{2}+\frac{1}{3!}-\frac{1}{4}+\cdots -\frac{1}{2n}+\frac{1}{(2n+1)!}-\cdots$は負の無限大に発散する.

(vii)

$x > 0$で$x >\log (1+x)$である.

$n > 0$で$x\rightarrow n^{\frac{1}{4}}$として
\[n^\frac{1}{4} > \log (1+n^\frac{1}{4}) >\log n^\frac{1}{4} = \frac{1}{4}\log n\]
$a_n = (-1)^n\frac{\log n}{\sqrt{n}}$とする.

$b_n=a_{2n+2}+a_{2n+3}$
とする.

\begin{align*}
|b_n| &= |\frac{\log (2n+2)}{\sqrt{2n+2}}-\frac{\log (2n+3)}{\sqrt{2n+3}}|\\
&=|\log (2n+2) (\frac{1}{\sqrt{2n+2}}-\frac{1}{\sqrt{2n+3}}) - \frac{\log \frac{2n+3}{2n+2}}{\sqrt{2n+3}}|\\
&\leqq |\log (2n+2) (\frac{1}{\sqrt{2n+2}}-\frac{1}{\sqrt{2n+3}})| +| \frac{\log \frac{2n+3}{2n+2}}{\sqrt{2n+3}}|\\
&= |\frac{\log (2n+2)}{\sqrt{2n+2}\sqrt{2n+3}(\sqrt{2n+2}+\sqrt{2n+3})}| +|\frac{\log (1+\frac{1}{2n+2})}{\sqrt{2n+3}}|\\
& < |\frac{4(2n+2)^{\frac{1}{4}}}{\sqrt{2n+2}\sqrt{2n+3}(\sqrt{2n+2}+\sqrt{2n+3})}|+|\frac{1}{(2n+2)\sqrt{2n+3}}|\\
& < 2(2n+2)^{-\frac{5}{4}}+(2n+2)^{-\frac{3}{2}}<\frac{2}{n^\frac{5}{4}}+\frac{1}{n^\frac{3}{2}}
\end{align*}
定理V 2.5より$\sum \frac{1}{n^{\frac{5}{4}}},\sum \frac{1}{n^{\frac{3}{2}}}$は収束する.

命題5.3 1)より$\sum \frac{2}{n^\frac{5}{4}}+\frac{1}{n^\frac{3}{2}}$も収束し定理5.5より$b_n$は絶対収束し収束する.

$\sum b_n = b$とする.

$\epsilon > 0$に対して$n\geqq M_1$で$\displaystyle|\sum_{k=1}^nb_k-b|<\frac{\epsilon}{2}$となる$M_1\in\mathbb{N}$が存在する.

$n>0$で
\[|(-1)^n\frac{\log n}{\sqrt{n}}| = \frac{\log n}{\sqrt{n}} < \frac{n^{\frac{1}{4}}}{\sqrt{n}}=n^{-\frac{1}{4}}\xrightarrow[n\to \infty]{}0\]
より$n\geqq M_2$で$|(-1)^n\frac{\log n}{\sqrt{n}}|<\frac{\epsilon}{2}$となる$M_2\in\mathbb{N}$が存在する.

$n\geqq 2M_1+3$で$n$が奇数の時
\[|\sum_{k=1}^na_k - b|=|\sum_{k=0}^{\frac{n-3}{2}}b_k - b| < \frac{\epsilon}{2}<\epsilon\]
$n\geqq \max(2M_1+4,M_2)$で$n$が偶数の時
\[|\sum_{k=1}^na_k - b|=|\sum_{k=0}^{\frac{n-4}{2}}b_k + (-1)^n\frac{\log n}{\sqrt{n}}- b| \leqq |\sum_{k=0}^{\frac{n-4}{2}}b_k- b|+|(-1)^n\frac{\log n}{\sqrt{n}}| <\frac{\epsilon}{2}+\frac{\epsilon}{2} = \epsilon\]

以上より$n \geqq \max(2M_1+4,M_2)$で$|\sum a_n - b|<\epsilon$で$\sum a_n$は$b$に収束する.

(viii)

二項定理より$n \geqq 1$で
\[(1+\frac{1}{n})^n =\sum_{k=0}^n\frac{\ _nC_k}{n^k} \geqq \frac{\ _nC_0}{n^0}+\frac{\ _nC_1}{n^1} = 2\]
よって
\[\frac{(1+n)^n}{n^{n+1}} = \frac{1}{n}(1+\frac{1}{n})^n \geqq \frac{2}{n}\]
例4より$\displaystyle \sum_{n=1}^\infty \frac{1}{n}$は発散するので定理5.5より$\sum \frac{(1+n)^n}{n^{n+1}}$は発散する.

(ix)

$a_n=\frac{3\cdot 5\cdots (2n+1)}{5\cdot 10\cdots 5n}$とする.
\[\frac{a_{n+1}}{a_n}=\frac{2n+3}{5n+5}=\frac{2+\frac{3}{n}}{5+\frac{5}{n}}\xrightarrow[n\to\infty]{}\frac{2}{5}<1\]
よって定理5.7より$\sum a_n$は収束する.

(x)

二項定理より$n \geqq 1$で
\[(1+\frac{1}{n})^n =\sum_{k=0}^n\frac{\ _nC_k}{n^k} \geqq \frac{\ _nC_0}{n^0}+\frac{\ _nC_1}{n^1} = 2\]
よって
\[(\frac{n}{n+1})^{n^2}=((1+\frac{1}{n})^n)^{-n}\leqq 2^{-n}\]
また
\[\sum_{k=1}^n 2^{-k}=\frac{1}{2}\frac{1-(\frac{1}{2})^n}{1-\frac{1}{2}} =1-\frac{1}{2}^{n}\xrightarrow[n\to\infty]{}1\]
で収束する.よって定理5.5より$\sum (\frac{n}{n+1})^{n^2}$は収束する.

(xi)

$n>e^2$で$n>1$より
\[\frac{1}{n^{\log n}} < \frac{1}{n^2}\]
例7より$\displaystyle \sum_{n=1}^\infty \frac{1}{n^2}$は収束するので定理5.5より$\sum \frac{1}{n^{\log n}}$は収束する.

3)

$\sum a_n$が絶対収束するとき$\displaystyle \lim_{n\to\infty}a_n = 0$だ.

よって$M\in\mathbb{N}$が存在し
\[n\geqq M \Rightarrow |a_n| < 1\]
$n\geqq M$で
\[|a_n^2|<|a_n|\]
なので定理5.5より$\sum |a_n|$が収束するとき$\sum |a_n^2|$も収束する.つまり$\sum a_n$が絶対収束するとき$\sum a_n^2$も絶対収束する.

4)

\[\sum a_n, a_{2n-1}=\frac{1}{\sqrt{n}}, a_{2n}=-\frac{1}{\sqrt{n}}\]
が条件を満たすことを示す.

$\displaystyle \lim_{n\to\infty}\frac{1}{\sqrt{n}}=0$なので$\epsilon > 0$に対して$M\in \mathbb{N}$が存在して
\[n\geqq M\Rightarrow |\frac{1}{\sqrt{n}}| < \epsilon\]
$n$が偶数のとき$\displaystyle \sum_{k=1}^n a_k = 0$

$n$が奇数のとき$n \geqq 2M+1$で
\[|\sum_{k=1}^n a_k| = \frac{1}{\sqrt{\frac{n-1}{2}}} < \epsilon\]
以上より$n\geqq 2M+1$で$\displaystyle |\sum_{k=1}^n a_k| < \epsilon$

より$\sum a_n$は収束する.

$b_n = a_n^2$とする.$n$が偶数のとき
\[\sum_{k=1}^n b_n = \sum_{k=1}^\frac{n}{2} \frac{2}{k}\]
例4よりこれは発散する.

部分列が発散するので$\sum b_n$も発散する.

以上よりこれが例となる.

5)

$\displaystyle\lim_{n\to\infty} \frac{a_n}{b_n}=c$なので
$n\geqq M$で$|\frac{a_n}{b_n}-c|< c$となる$M\in\mathbb{N}$が存在する.

$b_n >0$なので$n\geqq M$で
\[\frac{a_n}{b_n}<2c \Leftrightarrow a_n < 2cb_n\]
$\sum b_n$が収束するとき命題5.3より$\sum 2cb_n$も収束し定理5.5より$\sum a_n$も収束する.

$c\neq 0,+\infty$なので$\displaystyle\lim_{n\to\infty}
\frac{b_n}{a_n}=\frac{1}{c}$で$\frac{1}{c}\neq 0,+\infty$

より同様にして$\sum a_n$が収束するとき$\sum b_n$も収束する.

以上より$\sum a_n,\sum b_n$は同時に収束,発散する.

6)
\[\frac{1-(-x^2)^n}{1+x^2}=\frac{1-(-x^2)^n}{1-(-x^2)} =\sum_{k=0}^{n-1} (-x^2)^k\]
$I_n=\int_0^1\frac{(-x^2)^n}{1+x^2}dx$とすると両辺$0\sim 1$で積分し
\[\int_0^1\frac{1}{1+x^2}dx-I_n=\sum_{k=0}^{n-1} (-1)^k\frac{1}{2k+1}\]
$0<x<1$で$1+x^2>1$なので$0\sim 1$で積分し
\[|I_n| \leqq \int_0^1|\frac{x^{2n}}{1+x^2}|dx<\int_0^1|x^{2n}|dx=\frac{1}{2n+1}\xrightarrow[n\to\infty]{}0\]
よって$\displaystyle \lim_{n\to\infty}(\sum_{k=0}^n (-1)^k\frac{1}{2k+1}-\int_0^1\frac{1}{1+x^2}dx)=0\Leftrightarrow \lim_{n\to\infty}\sum_{k=0}^n (-1)^k\frac{1}{2k+1}=\int_0^1\frac{1}{1+x^2}dx$

$x=\tan \theta$とすると$dx = (\tan^2\theta +1)d\theta$で$x:0\to 1$で$\theta :0\to \frac{\pi}{4}$なので
\[\int_0^1\frac{1}{1+x^2}dx = \int_0^\frac{\pi}{4}\frac{1+\tan^2\theta}{1+\tan^2\theta}d\theta = \int_0^\frac{\pi}{4}1d\theta = \frac{\pi}{4}\]
より$(与式)=\frac{\pi}{4}$

7)

(i)

$|x|\leqq 1$で
\[\frac{1}{1+2nx^{2n}} \geqq \frac{1}{1+2n}>\frac{1}{2} \frac{1}{1+n}\]
例4より$\displaystyle \sum_{n=1}^\infty \frac{1}{n}$は発散するので定理5.5より$\sum \frac{1}{1+2nx^{2n}}$は発散する.

$|x| > 1$で
\[\frac{1+2nx^{2n}}{1+2(n+1)x^{2(n+1)}}=\frac{\frac{1}{2nx^{2n}}+1}{\frac{1}{2nx^{2n}}+(1+\frac{1}{n})x^2}\xrightarrow[n\to\infty]{}\frac{1}{x^2} < 1\]
より定理5.7より$\sum \frac{1}{1+2nx^{2n}}$は収束する.

より条件は$|x|>1 \Leftrightarrow x<-1または1<x$

(ii)

$a_n = \frac{1\cdot 3 \cdots (2n-1)}{2\cdot 4\cdots 2n}(1-x^2)^n$
\[\frac{|a_{n+1}|}{|a_n|}=\frac{2n+1}{2n+2}|1-x^2|=\frac{2+\frac{1}{n}}{2+\frac{2}{n}}|1-x^2|\xrightarrow[n\to\infty]{}\frac{2}{2}|1-x^2|=|1-x^2|\]
よって$|1-x^2|< 1$で$\sum a_n$は絶対収束し収束する.
\[\frac{2\cdot 4 \cdots (2n)}{3\cdot 5\cdots (2n+1)} < 1\]の両辺に$\frac{1\cdot 3 \cdots (2n-1)}{2\cdot 4\cdots 2n}$を掛けて
\[\frac{1}{2n+1} < \frac{1\cdot 3 \cdots (2n-1)}{2\cdot 4\cdots 2n}\]
なので$|1-x^2|>1$で
\[|a_n|\geqq \frac{|1-x^2|^n}{2n+1}\]
二項定理より$n\geqq 2$で
\[|1-x^2|^n = ((|1-x^2|-1)+1)^n = \sum_{k=0}^n \ _nC_k (|1-x^2|-1)^k > 
\ _nC_2 (|1-x^2|-1)^2 = \frac{n(n-1)}{2}(|1-x^2|-1)^2
\]
よって
\[|a_n| > \frac{n(n-1)}{4n+2}(|1-x^2|-1)^2 \xrightarrow[n\to\infty]{} \infty\]
で$\sum a_n$は発散する.

$x=0$のとき$a_n>0$で
\[\frac{a_{n+1}}{a_n}=\frac{2n+1}{2n+2}=1-\frac{1}{2n+2}>1-\frac{1}{n+1}=\frac{n}{n+1}\]
例4より$\displaystyle \sum_{n=1}^\infty \frac{1}{n}$は発散するので$\sum a_n$も発散する.

$1-x^2 = -1$のときを考える.

$b_n=a_{2n}+a_{2n+1}$とする.
\[b_n = \frac{1\cdot 3 \cdots (4n-1)}{2\cdot 4\cdots 4n}\frac{1}{4n+2}\]
で$b_n>0$である.
\[\frac{b_{n+1}}{b_n}=\frac{(4n+1)(4n+3)}{(4n+6)(4n+4)}\]
\begin{align*}
&(\frac{4n+1}{4n+4})^4=(1-\frac{3}{4n+4})^4\\
&< (1-\frac{3}{4n+16})(1-\frac{3}{4n+13})(1-\frac{3}{4n+10})(1-\frac{3}{4n+7})\\
&= \frac{4n+13}{4n+16}\frac{4n+10}{4n+13}\frac{4n+7}{4n+10}\frac{4n+4}{4n+7}=\frac{n+1}{n+4}\\
&=\frac{n+1}{n+2}\frac{n+2}{n+3}\frac{n+3}{n+4}\\
&=(1-\frac{1}{n+2})(1-\frac{1}{n+3})(1-\frac{1}{n+4})\\
&< (1-\frac{1}{n+4})^3=(\frac{n+3}{n+4})^3
\end{align*}
同様にして
\begin{align*}
&(\frac{4n+3}{4n+6})^4=(1-\frac{3}{4n+6})^4\\
&< (1-\frac{3}{4n+16})(1-\frac{3}{4n+13})(1-\frac{3}{4n+10})(1-\frac{3}{4n+7})\\
&<(\frac{n+3}{n+4})^3
\end{align*}
よって
\[\frac{b_{n+1}}{b_n}<(\frac{n+3}{n+4})^\frac{3}{2}\]
定理V 2.5より$\frac{1}{n^{\frac{3}{2}}}$は収束するので$\frac{1}{(n+3)^{\frac{3}{2}}}$も収束する.よって定理5.5より$\sum b_n$も収束する.

$\sum b_n=b$とする.

$4k^2 > 4k^2-1\Leftrightarrow \frac{2k}{2k+1}>\frac{2k-1}{2k}$なので$k=1,\cdots ,n$で掛けて
\[\frac{2\cdot 4 \cdots (2n)}{3\cdot 5\cdots (2n+1)}
 > \frac{1\cdot 3 \cdots (2n-1)}{2\cdot 4\cdots 2n} \]
両辺に$\frac{1\cdot 3 \cdots (2n-1)}{2\cdot 4\cdots 2n}$を掛けて
\[\frac{1}{2n+1} > (\frac{1\cdot 3 \cdots (2n-1)}{2\cdot 4\cdots 2n})^2\]
よって
\[\frac{1\cdot 3 \cdots (2n-1)}{2\cdot 4\cdots 2n} < \frac{1}{\sqrt{2n+1}}\xrightarrow[n\to\infty]{}0\]
より$\displaystyle \lim_{n\to\infty}|a_n| = 0$

$n$が奇数のとき
\[\sum_{k=0}^na_k = \sum_{k=0}^{\frac{n-1}{2}}b_k\xrightarrow[n\to\infty]{}b\]
$n$が偶数のとき
\[\sum_{k=0}^na_k = a_n+\sum_{k=0}^{\frac{n-2}{2}}b_n\xrightarrow[n\to\infty]{}0+b=b\]
よりいずれの場合も$b$に収束するので$\sum a_n$は収束する.

以上より収束する条件は$-\sqrt{2}\leqq x < 0$または$0<x\leqq\sqrt{2}$

8)

$S_n$を十進数表示で$9$が出てこない$n$桁の自然数の集合とし$s_{n,i}$を$S_n$の要素で$1$の位が$i$のものとする.

$A=S_n,s_{n,i}$に対し
\[
sum(A) = \sum_{k\in A}\frac{1}{k}
\]
とする.

$\sum sum(S_n)$は与えられた級数の部分列.与えられた級数が正項級数なことに注意する.

$sum(S_1)=1+\frac{1}{2}+\cdots + \frac{1}{8}=\frac{2283}{840}$

$n\geqq 2$で$i=0,\cdots ,8$で
\[sum(s_{n,i})=\sum_{k\in S_{n-1}}
\frac{1}{10k+i}\leqq \sum_{k\in S_{n-1}}\frac{1}{10k}=\frac{1}{10}sum(S_{n-1})\]
よって
\[sum(S_n)=\sum_{i=0}^8sum(s_{n-1,i})\leqq \frac{9}{10}sum(S_{n-1})\]
これを繰り返し使い$sum(S_n)\leqq (\frac{9}{10})^{n-1}sum(S_1)$
\[
\sum_{k=1}^n sum(S_k)\leqq sum(S_1) \sum_{k=1}^n (\frac{9}{10})^{k-1}= sum(S_1)\frac{1-(\frac{9}{10})^n}{1-\frac{9}{10}}<10sum(S_1)
\]
よって$\sum sum(S_n)$は収束して収束する値$s$は
\[s\leqq 10\frac{2283}{840} < 80\]
よって与えられた級数は収束して収束する値は80より小さい.
\end{document}
